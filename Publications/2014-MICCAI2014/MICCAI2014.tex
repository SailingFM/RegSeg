\documentclass{llncs}

\usepackage[acronym,nowarn]{glossaries}



\title{Active contours-based registration method for correction of susceptibility distortion in MR images}
\titlerunning{AC-based registration for EPI distortion correction}

\author{Oscar~Esteban$^{*}$,
        Alessandro~Daducci,
        Meritxell~Bach-Cuadra,
        Jean-Philippe~Thiran,
        Andr\'es~Santos,
        and~Dominique~Zosso%
}

\newacronym{mri}{MRI}{magnetic resonance imaging}
\newacronym{dmri}{dMRI}{diffusion MRI}
\newacronym{epi}{EPI}{echo planar imaging}
\newacronym{t1}{T1w}{T1-weighted}
\newacronym{t2}{T2w}{T2-weighted}
\newacronym{acwe}{ACWE}{active contours without edges}
\newacronym{psf}{PSF}{point spread function}
\newacronym[longplural=regions of interest]{roi}{ROI}{region of interest}
\newacronym{wm}{WM}{white matter}

% make sure this is after hyperref if present
\makeglossary
\glsdisablehyper

\begin{document}
\maketitle

\begin{abstract}
We propose an active contours without edges \mbox{-based} registration method
  to correct for the distortions that typically appear in
  images acquired using echo planar imaging (EPI) schemes.
These distortions produce a geometrical deformation characteristic of the images,
  and signal loss at affected voxels.
We validate the method on a small set of digital phantoms.
We report quantitative evaluation results on 20 magnetic resonance imaging (MRI) 
  datasets of healthy subjects warped with deformation maps derived from real 
  diffusion MRI (dMRI) datasets.
Finally, we cross-compare results versus an alternative volume-based registration
  method.
\end{abstract}

\begin{keywords}
diffusion MRI,~susceptibility~distortion,~segmentation,~registration,~parcellation,~shape-prior.
\end{keywords}

\section{Introduction}\label{sec:intro}
\Gls*{dmri} is a widely used family of \gls*{mri} techniques which
  accounts for growing interest in its applications to structural
  connectivity analysis of the brain at the macro-scale.
From \gls*{dmri} data, it is possible to derive the local axonal structure
  at each imaging voxel, and estimate a whole-brain mapping of fiber
  tracts represented by trajectories --and/or probability maps-- reconstructed
  from the local information \cite{craddock_imaging_2013}.
Currently, the connectome extraction and analysis relies on a large
  chain of sophisticated computational methods including acquisition,
  reconstruction, modeling and model fitting, image segmentation 
  and registration, fibre tracking, connectivity mapping, visualization,
  statistical analyses and inference.
The artifact is generally present in \gls*{mri} using \gls*{epi} schemes,
  and caused by variations on susceptibility at tissue interfaces.
A major geometrical feature of the distortions is directionality, as it 
  happens along the phase encoding direction.
  
The majority of methods published addressing the problem were
  proposed during the development of  functional \gls*{mri}. 
We refer the reader to Table S1 (Supplemental Material) for a
  extended survey.
However, the connectome extraction from \gls*{dmri} data requires
  a more accurate geometrical correction as it provides for
  structural information, and a highly precise \gls*{wm} segmentation
  to reliably perform tractography.
Susceptibility distortion has been recently proven a major source of
  variability of resulting tractographies and connectomes
  \cite{irfanoglu_effects_2012}.
  
In this work, we propose a simultaneous segmentation-registration
  method derived from \gls*{acwe} \cite{chan_active_2001}, that
  jointly overcomes the presented pitfalls.
A free deformation field is implemented with a multi-resolution 
  B-Spline diffeomorphic transform model, that concatenates a 
  certain number of levels --defined by an increasing amount of points
  forming the control grid--.
Optimization strategy is steepest descent, using shape-gradients
  \cite{jehan-besson_dream2s:_2003} to avoid a levelset implementation,
  and keep the model close to the original surfaces extracted from
  anatomically correct data.
These surfaces are obtained with \emph{FreeSurfer}\footnote{\url{http://...}}
  on \gls*{t1} images, to comply with the state of art procedures.
Regularization is performed on the deformation field, using anisotropic
  penalty terms that allow for favoring the phase-encoding direction.
To assess the validity of the method, we apply it to recover
  a synthetic B-Spline deformation from digital phantoms.
Finally, we use an existing evaluation framework \cite{esteban_phantom_2014}
  to distort real \gls*{mri} data with real deformation fields derived
  from \gls*{dmri} data.
We also compare our results with an alternative tool based on \gls*{t2}
  registration.
  
\section{Material and methods}\label{sec:methods}

Invertibility is ensured by keeping the maximum displacement under
  the threshold of $0.40 \times [ s_x, s_y, s_z ]$, where $s_i$ are
  the spacings between consecutive control points of each level
  \cite{rueckert_diffeomorphic_2006}.


\bibliography{references}
\bibliographystyle{splncs}

\end{document}
\begin{table}[h]
	\centering
	\rowcolors{2}{white}{lightgray}
	\footnotesize
    \begin{tabular}{>{\raggedright}p{0.15\textwidth}p{0.05\textwidth}p{0.7\textwidth}}
    \hline
    Author (1st) & Year & Method description \\ \hline
    \citeauthor{jezzard_correction_1995,jezzard_characterization_1998} & 1995 & First proposal of the fieldmap-based correction on \gls*{fmri}. \\
    \citeauthor{robson_measurement_1997} & 1997 & First proposal of the \gls*{psf} mapping method on
    \gls*{fmri}. \\
    \citeauthor{reber_correction_1998} & 1998 & Fieldmap-based methodology. \\
    \citeauthor{cordes_geometric_2000} & 2000 & First proposal of the reversed phase-encoding method
    for \gls*{fmri}, that acquires an extra \gls*{epi} image with opposed gradient increments. \\
    \citeauthor{chiou_simple_2000} & 2000 & First proposal of an additional technique, familiar to the reversed phase-encoding method. \\
	\citeauthor{kybic_unwarping_2000} & 2000 & Early proposal of the \gls*{t2}-B0 registration method (see \autoref{table:registration}). \\
	\citeauthor{studholme_accurate_2000} & 2000 & \gls*{t2}-B0 registration (see \autoref{table:registration}). \\
	\citeauthor{andersson_modeling_2001} & 2001 & Model fitting of the fieldmap. \\
	\citeauthor{zeng_image_2002} & 2002 & Comprehensive comparison between the fieldmap unwarping and the \gls*{psf} mapping methods. \\
	\citeauthor{zaitsev_point_2004} & 2004 & Application of the \gls*{psf} mapping method in parallel \gls*{epi}. \\
	\citeauthor{li_accounting_2007} & 2007 & \gls*{t2}-B0 registration (see \autoref{table:registration}).  \\
	\citeauthor{wu_comparison_2008} & 2008 & \gls*{t2}-B0 registration (see \autoref{table:registration}).  \\
	\citeauthor{hsu_correction_2009} & 2009 & Fieldmap unwarping, improved with model-based 
	\gls*{psf} mapping. \\
	\citeauthor{tao_variational_2009} & 2009 & \gls*{t2}-B0 registration (see \autoref{table:registration}).  \\
	\citeauthor{holland_efficient_2010} & 2010 & Reversed phase-encoding method. \\
	\citeauthor{andersson_comprehensive_2012} & 2012 & Probabilistic solution to the three main distortion roots (namely motion, eddy currents, and susceptibility) through modeling in a Gaussian process. \\ \hline
    \end{tabular}
    \caption[Susceptibility correction techniques]%
    {\textbf{Susceptibility correction techniques}. Historical review of the different methodological approaches.}
    \label{table:susceptibility}
\end{table}
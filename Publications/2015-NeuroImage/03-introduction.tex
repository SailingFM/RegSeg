% -*- root: 00-main.tex -*-
\section{Introduction}\label{sec:regseg-intro}
\Acrlong*{dmri} enables the mapping of microstructure \citep{basser_microstructural_1996}
  and connectivity \citep{craddock_imaging_2013} of the human brain \emph{in-vivo}.
It is generally acquired using \gls*{epi} schemes, since they are very fast at
  scanning a large sequence of images called \glspl*{dwi}.
Each \gls*{dwi} is sensitized with a gradient to probe proton diffusion in a certain
  orientation.
Subsequent processing involves describing the local microstructure with one of the available
  models, which range from the early \gls*{dti} proposed by \cite{basser_microstructural_1996}
  to current models such as \gls*{amico}.
The microstructural map is then used to draw the preferential orientations of diffusion
  across the brain using tractography \citep{mori_threedimensional_1999}.
Finally, a graph representing the corresponding structural network is built using
  the regions of a cortical parcellation as nodes and the fiber paths found by
  tractography as edges \citep{hagmann_mapping_2008}.
The methodologies to solve reconstruction, tractography and network building
  require the delineation of the anatomy in the \gls*{dmri} space.
Moreover, current trends on reconstruction \citep{jeurissen_multitissue_2014} and
  tractography \citep{smith_anatomicallyconstrained_2012} are increasingly using
  structural information to improve the microstructural mapping and fiber-tracking.

Possibly, the earliest structural information incorporated to aid \gls*{dmri} processing
  is the \gls*{wm} mask used as a termination criteria for tractography.
The standardized procedure to obtain this mask was thresholding the \gls*{fa} map.
However, the mask and subsequent analyses are highly dependent on the 
  threshold that is chosen \citep{taoka_fractional_2009}.
To overcome the unreliability of \gls*{fa} thresholding, and to broaden
  \gls*{wm} segmentation to brain tissue segmentation, a large number of
  methods have been proposed using \glspl*{dwi}, the \lowb{}, and \gls*{dti}-derived
  scalar maps such as \gls*{fa}, \gls*{adc} and others \citep{zhukov_level_2003,
  rousson_level_2004,jonasson_segmentation_2005,hadjiprocopis_unbiased_2005,liu_brain_2007,
  lu_segmentation_2008,han_experimental_2009}.
However, the precise segmentation of \gls*{dmri} is difficult for several reasons.
First, \gls{dmri} images have a resolution that is much lower
  than that of the imaged microstructural features.
Therefore, voxels located in structural discontinuities are affected by partial
  voluming of the signal sources.
Second, the extremely low \gls*{snr} and the high dimensionality of the \glspl*{dwi} prevent
  their direct use in segmentation.
Third, the low contrast between \gls*{gm} and \gls*{wm} in the \lowb{} volume also makes
  it unsuitable for brain tissue segmentation.

An alternative route to segmentation in \gls*{dmri} space is the mapping of the
  structural information extracted from anatomical MR images, such as \gls*{t1}, using
  image registration techniques.
Generally, intra-subject registration of MR images of the brain involves only a linear
  mapping to compensate for head motion between scans.
However, \gls*{epi} introduces a geometrical distortion \citep{jezzard_correction_1995}
  that impedes the linear mapping from the structural space.
Numerous methods have been proposed to overcome this problem by incorporating information
  from extra MR acquisitions such as fieldmaps \citep{jezzard_correction_1995},
  \glspl*{dwi} with a different \gls*{pe} scheme \citep{cordes_geometric_2000,chiou_simple_2000},
  or \gls*{t2} images \citep{kybic_unwarping_2000,studholme_accurate_2000}.
These methods estimate the deformation field associated to \emph{\gls*{epi} distortions} and
  resample the \glspl*{dwi} onto a corrected \gls*{dmri} space.
The retrospective \gls*{epi} correction is an active field of research yielding frequent refinements
  and combinations of the original methods, such as
  \citep{holland_efficient_2010,andersson_comprehensive_2012,
  irfanoglu_drbuddi_2015}.
A standardized method to solve the remaining linear mapping between the corrected-\gls*{dmri}
  and the structural spaces is \emph{bbregister} \citep{greve_accurate_2009}.

Here, we present a segmentation and surface-to-volume registration method
  called \regseg{}, and show its usefulness in mapping anatomical information from structural
  space into native \gls*{dmri} space to aid subsequent processing steps
  (reconstruction, tractography and network building using a cortical parcellation).
The underlying hypothesis is that the registration and segmentation problems
  in \gls*{dmri} can be solved simultaneously.
To implement \regseg{} we first establish an active-contours without edges
  \citep{chan_active_2001} segmentation framework.
A specific set of reference surfaces extracted from the same subject initialize
  the 3D active contours, which evolve searching for homogeneous regions in the multivariate
  target-image.
We apply \regseg{} to segment \gls*{dmri} data by mapping
  a set of nested surfaces extracted from a structural image (e.g. \gls*{t1}) to
  a bivariate target-volume comprehending the \gls*{fa} and \gls*{adc} maps.
The evolution of the surfaces is supported by a B-spline basis, optimized
  iteratively using a descent approach driven by shape-gradients
  \citep{besson_dream2s_2003,herbulot_segmentation_2006}.
Therefore, \regseg{} establishes a registration framework that actually
  deals with the nonlinear warping induced by \gls*{epi} distortions.
\Regseg{} integrates the benefits of segmentation and registration methods together and
  exploits the multivariate nature of \gls*{dmri} data to contribute in the proposed
  application on three key aspects:
  1) the surfaces are typically extracted from the \gls*{t1} image of the same subject,
    therefore \regseg{} does not require additional MR acquisitions to the minimal
    \gls*{dmri} protocol in order to estimate the deformation field;
  2) \revcomment[R\#1-C2]{%
    alternatively to the typical design of the processing flow, the information from the
    reference \gls*{t1} can be precisely mapped onto the distorted \gls*{dmri} space,
    avoiding the interpolation of the \glspl*{dwi} required by unwarping the diffusion data; and} 
%    depending on the application, the resampling of the \glspl*{dwi}
%    introduced by the correction method can be avoided, either by performing the posterior
%    processing on the native \gls*{dmri} space or applying the unwarping on the tractography
%    itself; and
  3) \regseg{} increases the geometrical accuracy of the overall process.
In this paper, we first verify the functionality of the method and the \regseg{}
  implementation using a set of digital phantoms, demonstrating the
  \revcomment[R\#3-C10]{subvoxel} accuracy in registration.
Then, we evaluate \regseg{} on real \gls*{dmri} datasets, using a derivation of our
  instrumentation framework \citep{esteban_simulationbased_2014} which simulates
  known and realistic \gls*{epi} distortions.
We also compare \regseg{} and a nonlinear registration method to map the \lowb{} to
  the corresponding \gls*{t2} image of the same subject.
This approach is the first step of the above-mentioned \gls*{t2b} correction
  methods.
We reproduce the settings and implementation of a widely used diffusion
  processing software \citep[\emph{ExploreDTI},][]{leemans_exploredti_2009}.
With this comparison, we demonstrate how \regseg{} achieves higher accuracy with the
  simultaneous registration and segmentation process.

% -*- root: 00-main.tex -*-
\section{Introduction}%
\label{sec:introduction}
Image registration is the process to find a spatial mapping that aligns the information available
  in two different coordinate systems.
In their comprehensive survey, \cite{sotiras_deformable_2013} illustrated the wide variety
  of existing methodologies, and classified them by their three principal building blocks:
  the matching criteria to evaluate the proximity of the solution,
  the deformation model that may implement theoretical properties of the mapping
  (i.e. linearity, elasticity, viscosity, etc.), and the optimization method.
The matching criteria can be defined as a function on the intensities of voxels of the
  two images, the alignment of spatial features derived from the corresponding objects in
  the spaces to be registered, or use both intensity and spatial informations.
In \citep{sotiras_deformable_2013} these methods are referred to as \emph{iconic}, \emph{geometric},
  and \emph{hybrid}, respectively.
When the matching criteria is defined over an optimizable partition of one or both reference an
  moving spaces, we face a hybrid method which simultaneously segments and registers images.
An early integration of segmentation and registration by \cite{bertalmio_morphing_2000} proposed
  a sequential deformation of active contours for object tracking in series of 2D images.
Shortly, \cite{yezzi_variational_2003} presented the first method including a full solution to
  the registration problem with an affine transformation supporting the mapping.
\cite{vemuri_joint_2003} proposed an atlas-based registration framework using level sets and only
  one \gls*{pde} for first time.
\cite{unal_coupled_2005} and later \cite{wang_joint_2006},
  extended the ``two \glspl*{pde}'' method of \cite{yezzi_variational_2003}
  to nonlinear registration implementing a free deformation field.
\cite{droske_mumfordshah_2009} reviewed the latter set of techniques, and proposed two different
  approaches to apply the Mumford-Shah functional \citep{mumford_optimal_1989} in simultaneous
  registration and segmentation, through the propagation of the deformation field from
  the contours to the whole image definition.
\cite{greve_accurate_2009} presented a widely used registration method called \emph{bbregister},
  included in \emph{FreeSurfer} \citep{fischl_freesurfer_2012}.
Their framework performs robust registration of brain surfaces into the intensity information
  of a target \gls*{mri}, using an affine transformation and active contours.
Recently, \cite{guyader_combined_2011} proposed a simultaneous segmentation and
  registration method using level sets and a nonlinear elasticity smoother on the
  displacement vector field, which preserves topology even with very large deformations.
Finally, \cite{gorthi_active_2011} extended the existing methodologies using a multiphase
  level-set function for the registration of several active contours, in the application
  of atlas-based segmentation.
Alternatively to the historical use of active contours, some Bayesian approaches
  have been proposed as well \citep{wyatt_map_2003,pohl_bayesian_2006,gass_simultaneous_2014}.

We propose \emph{regseg}, an active-contours driven registration method.
Then, we demonstrate its aptness on the correction of the nonlinear distortions typically
  presented by \gls*{dmri} data of the brain.
This problem is briefly described in \autoref{sec:distortions}.
The general overview of our method, along with the contributions with respect 
  to the current state of art are summarized in \autoref{sec:contributions}.
In \autoref{sec:methods_map}, we start from a probabilistic framework to finally show its duality
  with respect to the active-contours route.
Following sections
  introduce the specific features of \emph{regseg} (\autoref{sec:numerical_implementation}),
  state the evaluation experiments (\autoref{sec:experiments_evaluation}),
  and describe the synthetic and real data involved in this study (\autoref{sec:datasets}).
In \autoref{sec:results} we describe our findings to support the usefulness of \emph{regseg}
  in the described application, compared to a well-established tool.
The interpretation of the results and prospects is addressed in \autoref{sec:discussion}.



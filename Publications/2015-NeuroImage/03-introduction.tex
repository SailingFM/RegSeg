% -*- root: 00-main.tex -*-
\section{Introduction}\label{sec:introduction}
The accurate delineation of \gls*{wm} in \gls*{dmri} and the fusion of prior
  anatomical information extracted from a \gls*{t1} image of the same subject
  are crucial in a range of applications based on tractography, such as
  the extraction of structural connectivity \citep{craddock_imaging_2013} or
  tract-based spatial statistics \citep{smith_tractbased_2006}.
However, segmenting \gls*{dmri} data precisely and coregistration of anatomical
  images are difficult for several reasons.
First, \gls{dmri} images suffer from partial voluming of several tissues in 
  interfacing voxels due to the low resolution of images (generally around
  \isores{2.2}mm).
Second, \gls*{dmri} schemes probe the diffusion process within the brain in 
  many angles called \glspl*{dwi}, completed by one or more baseline (\emph{b0}) 
  volumes without directional gradients.
The extremely low \gls*{snr} and high dimensionality of \glspl*{dwi} disable their
  use for direct segmentation.
The contrast between \gls*{gm} and \gls*{wm} in the \emph{b0} volumes is not suitable for 
  segmentation either.
Finally, \gls*{dmri} images are acquired using \gls*{epi} to speed up acquisition
  at the cost of presenting geometrical distortion and 
  signal losses \citep{jezzard_correction_1995} that hinder the segmentation by
  registering an anatomical image as source of same-subject prior information.
The artifact has been proven to cause an important impact on the anatomy extracted
  from \gls*{dmri} \citep{irfanoglu_effects_2012}, particularly in certain fiber bundles
  when used in whole-brain tractography.

Early attempts to delineate the \gls*{wm} proceeded by thresholding the 
  \gls*{fa}\footnote{the \gls*{fa} is a scalar parameter of diffusion derived from
  the \gls*{dmri} data.} map at values around 0.7.
The approach is unreliable for the high variability of \gls*{fa} and its
  decay with the increase of incoherence between fibers.
\cite{zhukov_level_2003} proposed active contours with edges represented
  by level sets, evolving on directionally invariant scalar maps.
\cite{rousson_level_2004} successfully segmented the corpus callosum with
  region-based level-sets on the eigenvalues of tensors fitted on the
  \gls*{dmri} data.
A large body of work targets the definition of appropriate features to segmentation,
  such as the 5D representation proposed by \cite{jonasson_segmentation_2005}.
Other efforts include mixture models on sets of directionally invariant maps
  \citep{liu_brain_2007}, iterative \citep{hadjiprocopis_unbiased_2005} and
  hierarchical \citep{lu_segmentation_2008} clusterings,
  and graph-cuts \citep{han_experimental_2009},
  and volume fraction modeling \citep{kumazawa_improvement_2013}.
 
To solve registration, \cite{saad_new_2009} used Pearson correlation to perform linear
  alignment of \gls*{t1} and \emph{b0}.
Another linear registration method is \emph{bbregister} \citep{greve_accurate_2009},
  that uses active contours extracted from the \gls*{t1}\footnote{using \emph{FreeSurfer}
  \citep{fischl_freesurfer_2012}} to look for intensity boundaries in the \emph{b0}
  image.
In practice, the target \emph{b0} does not present clear frontiers of intensity between
  interfacing tissues, what limits \emph{bbregister} to mapping the cortical surface only.
Since the distortions found in \gls*{dmri} are nonlinear, \emph{bbregister} excludes
  from the boundary search those regions typically warped by the artifact.
This tool has become the standard method for its proven robustness.
Nonlinear registration has been successfully performed between \gls*{t2} and \emph{b0}
  images for their similarity, but uniquely as a way to correct for distortions
  \citep{kybic_unwarping_2000,studholme_accurate_2000,wu_comparison_2008,tao_variational_2009}.
Further registration of \gls*{t1} and \gls*{t2} images was still required to map the anatomical
  information (and the \gls*{wm} segmentation) into \gls*{dmri} space.


To jointly solve registration and segmentation problems such as longitudinal object
  tracking \citep{paragios_level_2003} and atlas-based segmentation 
  \citep{gorthi_active_2011}, active contours have been successfully integrated 
  in image registration methods.
\cite{unal_coupled_2005} and later \cite{wang_joint_2006},
  extended an existing method using linear registration \citep{yezzi_variational_2003}
  to the nonlinear case implementing a free deformation field.
\cite{droske_mumfordshah_2009} reviewed the latter set of techniques, and proposed two different
  approaches to apply the Mumford-Shah functional \citep{mumford_optimal_1989} in simultaneous
  registration and segmentation, through the propagation of the deformation field from
  the contours to the whole image definition.
Recently, \cite{guyader_combined_2011} proposed a simultaneous segmentation and
  registration method using level sets and a nonlinear elasticity smoother on the
  displacement vector field, which preserves topology even with very large deformations.
Finally, \cite{gorthi_active_2011} extended the existing methodologies using a multiphase
  level-set function for the registration of several active contours, in the application
  of atlas-based segmentation.
  
The hypothesis under this research is that the registration and segmentation
  problems can be solved jointly, increasing the geometrical accuracy of the process.
We propose \emph{regseg} as a joint solution to distortion correction and segmentation,
  using active contours without edges \citep{chan_active_2001} that evolve driving a
  free-deformation field of B-Spline functions.
Optimization is performed using a descent strategy of shape-gradients
  \citep{herbulot_segmentation_2006,besson_dream2s_2003}, therefore \emph{regseg}
  does not implement level sets as most of the presented methods.
Given the properties of the distortion, \emph{regseg} includes the anisotropic regularization
  of displacement fields of \cite{nagel_investigation_1986}.

% While all the correction methods require an extra acquisition, \emph{regseg} is especially
%   indicated in scanning protocols without specific data for correction,
%   such as historical datasets.
Finally, we present an instrumentation framework to evaluate correction and
  segmentation pipelines using realistic distortions as ground truth.
With the integration of \emph{regseg} and an in-house implementation of the 
  \gls*{t2b} correction in the framework, we compare the performance
  of both solutions.

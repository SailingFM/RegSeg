% -*- root: 00-main.tex -*-
\section{Introduction}\label{sec:introduction}

\Gls*{dmri} data of the brain present severe distortions, directly related to the
  effort in minimizing acquisition time.
In order to quickly probe the diffusion process in sufficiently different orientations,
  \gls*{dmri} are acquired with \gls*{epi} sequences, which minimize the bandwidth of the \gls*{pe}
  direction.
This limitation originates a warping of the reconstructed image in regions affected by
  small deviations from the main magnetic field $B_0$.
One important source of these deviations is the discontinuity of magnetic susceptibility across
  the imaged object.
In fact, the distortion is the effect of the underlaying misregistration between correct and
  warped data.
Tissue/air boundaries show large steps of susceptibility and this artifact is
  particularly evident in the surroundings of the orbito-frontal lobe and the temporal
  bone of both hemispheres.
Several approaches have been proposed to estimate the distortion and correct
  for the artifact.
The first solution consisted on estimating the deviations from $B_0$ with field mapping
  techniques \citep{andersson_modeling_2001}.
A second family of methods require an extra \gls*{epi} acquisition, but switching
  the \gls*{pe} axis \citep{chiou_simple_2000} or reversing the direction of gradient \gls*{pe}
  increments \citep{cordes_geometric_2000,holland_efficient_2010}.
\cite{kybic_unwarping_2000} proposed nonlinear registration of the \emph{b0}
  image of the \gls*{dmri} to an undistorted \gls*{t2} image of
   the same subject.

The artifact has been proven to cause an important impact on the anatomy extracted
  from \gls*{dmri} \citep{irfanoglu_effects_2012}, particularly in certain fiber bundles
  when used in whole-brain tractography.
Currently, a broad range of applications make use of whole-brain tractography, such as
  connectome extraction \citep{craddock_imaging_2013} or tract-based spatial statistics
  \citep{smith_tractbased_2006}.
These applications generally require the segmentation of the \gls*{dmri} data, also
  hindered by distortions.
One way to identify the regions of interest is extracting them from an appropriate 
  \gls*{t1} of the same subject and map them into diffusion space,
  in a sort of atlas-based segmentation.

\paragraph{State of the art}\label{sec:state_of_art}
  Active contours have been successfully integrated in image registration methods
  for applications such as longitudinal object tracking \citep{paragios_level_2003} and
  atlas-based segmentation \citep{gorthi_active_2011}.
\cite{bertalmio_morphing_2000} proposed morphing active-contours for object tracking in 
  series of 2D images.
Shortly, \cite{yezzi_variational_2003} presented the first method including a full solution to
  the registration problem with an affine transformation supporting the mapping.
\cite{vemuri_joint_2003} proposed an atlas-based registration framework using level sets and only
  one \gls*{pde} for first time.
\cite{unal_coupled_2005} and later \cite{wang_joint_2006},
  extended the ``two \glspl*{pde}'' method of \cite{yezzi_variational_2003}
  to nonlinear registration implementing a free deformation field.
\cite{droske_mumfordshah_2009} reviewed the latter set of techniques, and proposed two different
  approaches to apply the Mumford-Shah functional \citep{mumford_optimal_1989} in simultaneous
  registration and segmentation, through the propagation of the deformation field from
  the contours to the whole image definition.
Recently, \cite{guyader_combined_2011} proposed a simultaneous segmentation and
  registration method using level sets and a nonlinear elasticity smoother on the
  displacement vector field, which preserves topology even with very large deformations.
Finally, \cite{gorthi_active_2011} extended the existing methodologies using a multiphase
  level-set function for the registration of several active contours, in the application
  of atlas-based segmentation.
  
\cite{greve_accurate_2009} presented a relevant registration method called \emph{bbregister},
  now included in \emph{FreeSurfer} \citep{fischl_freesurfer_2012}.
The method is a standard choice to map surfaces extracted in the \gls*{t1} image to the \emph{b0}
  image\footnote{The term \emph{b0} refers to one or more images within the
  \gls*{dmri} dataset that are acquired with a null or very low gradient intensity or
   \emph{b}-value.} of the \gls*{dmri} dataset with two limitations.
First, the transform is affine and therefore it does not cope with the nonlinear distortions.
Second, \emph{bbregister} looks for intensity steps in the image, using active contours with edges.
In practice, the target \emph{b0} does not present clear frontiers of intensity between
  interfacing tissues, what limits \emph{bbregister} to mapping the cortical surface only.

\paragraph{Contributions}\label{sec:contributions}
The hypothesis under this research is that the registration and segmentation
  problems can be solved jointly, increasing the geometrical accuracy of the process.
We propose \emph{regseg} as a joint solution to distortion correction and segmentation,
  using active contours without edges \citep{chan_active_2001} that evolve driving a
  free-deformation field of B-Spline functions.
Optimization is performed using a descent strategy of shape-gradients
  \citep{herbulot_segmentation_2006,besson_dream2s_2003}, therefore \emph{regseg}
  does not implement level sets as most of the presented methods.
Given the properties of the distortion, \emph{regseg} includes the anisotropic regularization
  of displacement fields of \cite{nagel_investigation_1986}.

While all the correction methods require an extra acquisition, \emph{regseg} is especially
  indicated in scanning protocols without specific data for correction,
  such as historical datasets.
Finally, we present an instrumentation framework to evaluate correction and
  segmentation pipelines using realistic distortions as ground truth.
With the integration of \emph{regseg} and an in-house implementation of the 
  \gls*{t2b} correction in the framework, we compare the performance
  of both solutions.

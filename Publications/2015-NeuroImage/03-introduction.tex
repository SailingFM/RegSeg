% -*- root: 00-main.tex -*-
\section{Introduction}
\label{sec:introduction}

\IEEEPARstart{N}{onlinear} registration of images is a challenging task ubiquitous
  in an endless number of image analysis applications.
The process aims to find a spatial mapping $U$ that aligns the information available
  in two different coordinate systems (generally called
  \emph{reference}, $R$, and \emph{moving}, $M$):
  \begin{align}
  U\colon R \subset \mathbb{R}^n &\to M \subset \mathbb{R}^n \notag\\
  \vec{r} &\mapsto \vec{r}' =\vec{r}+u(\vec{r}),
  \label{eq:transform}
  \end{align}
  where $\vec{r}$ denotes a position in the reference domain $R$, $\vec{r}'$ is
  its corresponding location in $M$, and $n$ the dimensionality of images.
Finally, $\vec{u} = u(\vec{r})$ is the displacement of every point with respect
  to the reference domain.
A comprehensive survey by \citeauthor{sotiras_deformable_2013} \citep{sotiras_deformable_2013}
  illustrates the wide variety of existing registration methodologies.
They classify the algorithms by the three principal building blocks that can be identified
  among methods: the matching criteria or similarity function to evaluate the proximity of
  the solution, the deformation model that may implement theoretical properties and constraints
  of the mapping in the selected application, and the optimization method or search strategy.
The most widespread matching criteria are probably the so-called \emph{iconic
  methods} \citep{sotiras_deformable_2013}.
These algorithms require the definition of an appropriate function on the voxel-wise information
  content of both images.
A second group, called \emph{geometric methods}, measure distances between features derived from
  corresponding objects defined in both spaces, such as landmarks, shapes, surfaces, etc.
Finally, \emph{hybrid methods} combine properties of the iconic and geometric methods.
As reviewed in \citep{sotiras_deformable_2013}, an important research effort has been devoted to
  study application-oriented matching criteria.


Currently uprising applications, such as the extraction of connectivity networks from
  diffusion \gls*{mri} \citep{craddock_imaging_2013}, demand for improved approaches
  that outperform existing techniques.
We propose the image registration -based solution to correct for the
  susceptibility distortions in \gls*{dmri} of the brain ,
  that is involved in the extraction of the structural connectivity map of the brain.
However they are not appropriate for all the registration problems
  \citep{greve_accurate_2009}.
They propose a \emph{hybrid method} \citep{sotiras_deformable_2013} for the
  intra-subject registration of \gls*{t1} \gls*{mri} and \gls*{dmri} images
  of the brain using a linear transformation model.
As in \citep{greve_accurate_2009}, we exploit the precise structures that
  can be readily extracted with widely-used tools at high resolution
  and map this prior information to low resolution data,
  in which the scale of the imaged structures is in the range of the image
  resolution or above.
Our approach differs with \citep{greve_accurate_2009} in two fundamental
  choices: 1) We use a nonlinear deformation model that enables the
  correction of the susceptibility-derived distortions
  \citep{jezzard_correction_1995} that \gls*{dmri} data typically present.
And 2) our matching criteria does not necessarily need for clear intensity
  steps at the tissue interfaces where surfaces must be fitted.

The hypothesis underpinning this research is that, in those situations,
  image registration can be reliably performed by searching for homogeneous
  regions in one image corresponding to a prior knowledge of the shape of objects in
  the moving coordinate system.


%{\color{red} {The main diverging points with respect to
%\citep{gorthi_active_2011} are: 1) there is no need for an explicit level set function
%$\Phi_G$, as we replace the level set gradient computation $N_{\Phi_G}$ with shape
%gradients \citep{besson_dream2s_2003,herbulot_segmentation_2006};
%2) regularization is also based on linear diffusion smoothing \citep{thirion_image_1998},
%but we replace the Gaussian filtering by other constraints
%studied in \citep{nagel_investigation_1986} to better the problem;
%3) optimization is applied in the spectral
%domain, observing anisotropic and inhomogeneous mappings along each direction.
%With respect \citep{guyader_combined_2011}, the main differences are
%the distance function, and the spectral solution to the optimization updates,
%as we shall cover in \autoref{sec:numerical_implementation}.}}



%In this paper we formulate the joint registration-segmentation
%problem as follows.
%such that the known contours in anatomical space $T$ optimally segment
%the diffusion space $D$.
%
%
%Whereas related \glspl*{adf} introduced in \autoref{sec:methods_background}
%make use of explicit level-set formulations to solve \eqref{eq:gradient_descent},
%we alternatively use \emph{shape-gradients}
%\cite{besson_dream2s_2003,herbulot_segmentation_2006}.

%
%Let us denote by $\vec{x}$ the voxel and $F(\vec{x}) = [ f_1, f_2, \ldots, f_N]^T$
%  its associated feature vector in the following.
%
%In our application, these surfaces are precise tissue interfaces of interest extracted
%  from a high-resolution, anatomically correct reference volume using
%  the well-established
%
%In this paper we propose a novel registration framework to simultaneously
%solving the segmentation, distortion and cortical parcellation challenges,
%by exploiting as strong shape-prior the detailed morphology extracted
%from high-resolution and anatomically correct \gls{mri}.
%Indeed, hereafter
%we assume this segmentation problem in anatomical images is reliably and
%accurately solved with readily available tools (e.g.
%\citep{fischl_freesurfer_2012}).
%After global alignment with \gls{t1} using existing approaches, the remaining
%spatial mismatch between anatomical and diffusion space is due to susceptibility
%distortions.
%Finally, we need to establish precise spatial correspondence between the
%surfaces in both spaces, including the tangential direction for parcellation.
%Therefore, we can reduce the problem to finding the differences of spatial
%distortion in between anatomical and \gls{dwi} space.
%We thus reformulate the segmentation problem as an inverse problem, where we
%seek for an underlying deformation field (the distortion) mapping
%from the structural space into the diffusion space, such that the structural
%contours segment optimally the \gls{dwi} data.
%In the process, the one-to-one
%correspondence between the contours in both spaces is guaranteed, and projection
%of parcellisation to \gls{dwi} space is implicit and consistent.
%
%We test our proposed joint segmentation-registration model on two different
%synthetic examples.
%The first example is a scalar sulcus model, where the
%\gls{csf}-\gls{gm} boundary particularly suffers from \gls{pve} and can only be
%segmented correctly thanks to the shape prior and its coupling with the inner,
%\gls{gm}-\gls{wm} boundary through the imposed deformation field regularity.
%The second case deals with more realistic \gls{dwi} data stemming from
%phantom simulations of a simplistic brain data.
%Again, we show that the
%proposed model successfully segments the \gls{dwi} data based on two derived
%scalar features, namely \gls{fa} and \gls{md}, while establishing an estimate
%of the dense distortion field.
%
%The rest of this paper is organized as follows.
%First, in \autoref{sec:methods}
%we introduce our proposed model for joint multivariate segmentation-registration.
%Then we provide a more detailed description of the data and experimental setup in
%\autoref{sec:experiments}.
%We present results in \autoref{sec:results} and conclude
%in \autoref{sec:conclusion}.
%
%
%The properties of the reconstructed tensors and derived scalar maps have
%been studied by \cite{ennis_orthogonal_2006}. Based on their
%findings, \gls{fa}~\eqref{eq:fa} and \gls{md}~\eqref{eq:md} are
%considered complementary features, and therefore we selected them for the
%energy model \eqref{eq:complete_energy} in driving the
%registration-segmentation process.
%Whereas \gls{fa} informs mainly about the \emph{shape} of diffusion,
%the \gls{md} is more related to the \emph{magnitude} of the process:
%
%\begin{align}
%\mathrm{FA} &= \sqrt{ \frac{1}{2}}\,\frac{\sqrt{ (\lambda_1 - \lambda_2)^2 + (\lambda_2 - \lambda_3)^2 + (\lambda_3 - \lambda_1)^2}}{\sqrt{ {\lambda_1}^2 + {\lambda_2}^2 + {\lambda_3}^2}} \label{eq:fa} \\
%\mathrm{MD} &= ( \lambda_1 + \lambda_2 + \lambda_3 ) / 3 \label{eq:md}
%\end{align}
%where $\lambda_i$ are the eigenvalues of the diffusion tensor
%associated with the diffusion signal $S(\vec{q})$. There exist
%two main reasons to justify their choice.
%First, they are well-understood and standardized in clinical routine.
%Second, together they contain most of the information that is
%usually extracted from the \gls{dwi}-derived scalar maps
%\cite{ennis_orthogonal_2006}.
%

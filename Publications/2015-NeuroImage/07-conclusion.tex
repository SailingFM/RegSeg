% -*- root: 00-main.tex -*-
\section*{Conclusion}
\label{sec:conclusion}

\Regseg{} is a variational framework to simultaneously segment and
  register 3D \gls*{dmri} data of the human brain, using within-subject
  anatomical information as reference.
The registration method segments the target multivariate image in several competing regions
  defined explicitly by their limiting surfaces.
The surfaces are active, and evolve on a free-form deformation field supported by B-Splines.
A descent optimization strategy is guided by shape gradients computed on the current partition
  of the target image.
\Regseg{} uses active contours without edges, and hence looks for
  homogeneous regions within the image.
We tested \regseg{} on digital phantoms simulating \gls*{t1} and \gls*{t2} \gls*{mri},
	warped with smooth random deformations.
The resulting misregistration of the contours was significantly lower than the image resolution
  of the phantoms.

We propose \regseg{} to simultaneously segment \gls*{dmri} data and register them to
  their corresponding \gls*{t1} image of the same subject.
Visual assessment of \regseg{} and cross-comparison with a widely-used technique demonstrated 
  its accuracy.
Moreover, \regseg{} does not require additional images to the minimal acquisition protocol
  that includes only \gls*{t1} and \gls*{dmri}.
Beyond the proposed application in \gls*{dmri} data, other potential uses of \regseg{} are
  atlas-based segmentation and tracking objects in time-series.


\section*{Availability and reproducibility statement}
\label{sec:availability}
We considered the reproducibility of our results a design requirement.
Therefore, we used real data fetched from a publicly available repository
  (the Human Connectome Project \citep{essen_human_2012}) and all the software
  involved in this paper is also publicly released.
\Regseg{} is implemented on top of ITK-4.6 (Insight Registration and 
  Segmentation Toolkit, \url{http://www.itk.org}).
The evaluation instruments (\autoref{fig:evworkflows}), are implemented using
  \emph{nipype} \citep{gorgolewski_nipype_2011} to assess their reproducibility.
All research elements (data, source code, figures, manuscript sources, etc.) in this paper
  are made publicly available under a unique package \citep{esteban_acweregistration_2015}.
% -*- root: 00-main.tex -*-
\renewcommand{\theequation}{A.\arabic{equation}}

\section*{Appendix}

\subsection*{Instant velocity of contours}
\label{app:shape_priors}
The computation of gradients at the locations of the active contours in the
  instant $t$ is based on the work of \cite{herbulot_segmentation_2006}.
Let $F(\vec{r})$ be an ``arbitrary'' function over the image domain
  $\Omega = \Omega_l \cup \Omega_m$ splitted in two regions $l$ and
  $m$, and $\Gamma_{l,m}$ a closed boundary between them
  as defined in \nameref{sec:methods_map}.
We now derive the domain integral w.r.t. $t$:

  \begin{align*}
  \frac{\partial}{\partial t} \int_\Omega F(\vec{r}) d\vec{r} &=
  \int_\Omega \frac{\partial}{\partial t}F(\vec{r}) d\vec{r} \notag\\
  &- \int_{\Gamma_{l,m}} F(\vec{r}) \left\langle \frac{\partial \Gamma_{l,m} }{\partial t},
  N_{\Gamma_{l,m}}\right\rangle d\vec{r},
  \end{align*}

  where $\left\langle\frac{\partial\Gamma_{l,m}}{\partial t}, N_{\Gamma_{l,m}}\right\rangle$ is
  the projection of the boundary movement on the unit inward normal $N_{\Gamma_{l,m}}$.
Assuming that the region descriptors $\{\mu_l, \Sigma_l\}$ vary slowly enough, we can consider
  that $\frac{\partial}{\partial t} F(\vec{r}) = 0$ and thus:

  \begin{equation*}
  \frac{\partial}{\partial t} \int_\Omega F(\vec{r}) d\vec{r} =
  - \int_{\Gamma_{l,m}} F(\vec{r}) \left\langle \frac{\partial \Gamma_{l,m} }{\partial t},
  N_{\Gamma_{l,m}}\right\rangle d\vec{r}
  \end{equation*}

  As inner ($\Omega_l$) and outer ($\Omega_m$) regions are competing and introducing
    the energy functional \eqref{eq:energy}:

    \begin{equation}
    \frac{\partial}{\partial t} \int_\Omega F(\vec{r}) d\vec{r} =
    - \int_{\Gamma_{l,m}} \left[ \mathcal{E}_{l}(\vec{r}) - \mathcal{E}_{m}(\vec{r}) \right]
    \left\langle \frac{\partial \Gamma_{l,m} }{\partial t},
  N_{\Gamma_{l,m}}\right\rangle d\vec{r},
  	\label{eq:shape_gradients}
    \end{equation}

    where $\mathcal{E}_l(\vec{r}) = \mathcal{D}^2_{l}(R(\vec{r}))$ for briefness.


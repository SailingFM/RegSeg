% -*- root: 00-main.tex -*-
\renewcommand{\theequation}{A.\arabic{equation}}
\renewcommand{\thesubsection}{Appendix \arabic{subsection}}

\section*{Appendix}

\subsection{Simplifying the regularization term}\label{app:reg_term}
The exponentials of the Thikonov regularization prior \eqref{eq:regseg-thikonov} have the general form
  $\vec{v}^T \mathbf{M} \vec{v}$.
If $\mathbf{M}$ is a $n \times n$ diagonal matrix such that $\mathbf{M} = \vec{m} \, \mathbf{I}_n$,
  then:

\begin{equation*}
\vec{v}^T \mathbf{M} \vec{v} = \vec{m} \cdot (\vec{v}^T \mathbf{I}_n \vec{v}) = \vec{m} \cdot \vec{v}^{\circ2},
\end{equation*}
  where we have introduced the Hadamard power notation\footnote{The Hadamard power of a matrix or a vector
  is the power of its elements $\mathbf{M}^{\circ p} = ({m_{ij}}^{p})$}.

In general, the anisotropy \revcomment[R\#3-C.31]{of the distortion field} is aligned with the 
  \revcomment[R\#3-C.31]{voxel coordinate system}, so
  $\mathbf{A}$ and $\mathbf{B}$ of \eqref{eq:regseg-energy} can be simplified to diagonal matrices
  \revcomment[R\#3-C.31]{to regularize the registration process}, such that
  $\mathbf{A}= \, \boldsymbol{\alpha}\,\vec{I}_n$ and
  $\mathbf{B}= \, \boldsymbol{\beta}\,\vec{I}_n$.
By substituting into equation \eqref{eq:regseg-energy}, we obtain:

  \begin{align}
  E(\vec{u}) &= \const + \, \underset{l}{\sum} \int_{\Omega_l} \mdist{f'}{l} \,d\vec{r} \,
  \ifthenelse{\boolean{review}}{+}{+ \notag\\ &+}
  \int_{\Omega} \frac12 \left[ \boldsymbol{\alpha} \cdot \vec{u}^{\circ2} + \boldsymbol{\beta} \cdot (\nabla \vec{u})^{\circ2} \right] \,d\vec{r}.
  \label{eq:regseg-app_energy}
  \end{align}

\subsection{Application of the shape-gradients}\label{app:shape_gradients}
\revcomment[R\#3-C.8]{%
The computation of gradients at the locations of the active contours in the
  instant $t$ is based on the work of \cite{herbulot_segmentation_2006}.
Let $F(\vec{r})$ be an ``arbitrary'' function over the image domain
  $\Omega = \Omega_l \cup \Omega_m$ split in two regions $l$ and
  $m$, and $\Gamma_{l,m}$ a closed boundary between them.
We now derive the domain integral w.r.t. $t$:

  \begin{equation}
  \frac{\partial}{\partial t} \int_\Omega F(\vec{r}) d\vec{r} =
  \int_\Omega \frac{\partial}{\partial t}F(\vec{r}) d\vec{r}
  - \int_{\Gamma_{l,m}} F(\vec{r}) \left\langle \frac{\partial \Gamma_{l,m} }{\partial t},
  N_{\Gamma_{l,m}}\right\rangle d\vec{r},
  \end{equation}
%
  where $\left\langle\frac{\partial\Gamma_{l,m}}{\partial t}, N_{\Gamma_{l,m}}\right\rangle$ is
  the projection of the boundary movement on the unit inward normal $N_{\Gamma_{l,m}}$.
Assuming that the region descriptors $\{\boldsymbol{\mu}_l, \boldsymbol{\Sigma}_l\}$ vary slowly enough, we can consider
  that $\frac{\partial}{\partial t} F(\vec{r}) = 0$ and thus:

  \begin{equation}
  \frac{\partial}{\partial t} \int_\Omega F(\vec{r}) d\vec{r} =
  - \int_{\Gamma_{l,m}} F(\vec{r}) \left\langle \frac{\partial \Gamma_{l,m} }{\partial t},
  N_{\Gamma_{l,m}}\right\rangle d\vec{r}.
  \label{eq:regseg-shape_gradients}
  \end{equation}

The equation \eqref{eq:regseg-shape_gradients} is discretized as follows.
First, the surface between limiting regions $l$ and $m$ ($\Gamma_{l,m}$) is explicitly represented by
  a discrete set of vertices $\vec{v}_i$, with $i \in \{0, \ldots, N_p -1 \}$.
Consequently, the inwards normal of the surface $N_{\Gamma_{l,m}}$ is represented by the discrete
  set of normals $\hat{\vec{n}}_i$ at each vertex of the mesh.
The resulting summation is, therefore, discrete and the integral operator is replaced by the sum:

  \begin{align}
  \frac{\partial}{\partial t} \int_\Omega F(\vec{r}) d\vec{r} &=
  \underbracket{\cancel{\int_\Omega \frac{\partial}{\partial t}F(\vec{r}) d\vec{r} }}_{\text{Functional's evolution}}
  - \underbracket{\int_{\Gamma_{l,m}} F(\vec{r}) \left\langle \frac{\partial \Gamma_{l,m}}{\partial t},
  N_{\Gamma_{l,m}}\right\rangle d\vec{r}}_{\text{Shape's evolution}} \notag \\
  & = - \underset{p}{\sum} \frac{1}{A_p} \underset{i}{\sum} \, a_i \, F(\vec{v}_i) \left\langle \underbracket{\frac{\partial \vec{v}_i}{\partial t}}_{\text{speed of }\vec{v}_i},
  \hat{\vec{n}}_{i}\right\rangle.
  \label{eq:regseg-shape_gradient_orig}
  \end{align}
where $a_i$ is the area corresponding to vertex $\vec{v}_i$, and $A_p = \sum_i a_i$ is the total area of surface $p$.
In the following, we will refer as $w_{p,i} = a_i / A_p $ to the area contribution of $\vec{v}_i$ to the
  total area of the surface it belongs to.
For simplicity, the sum over $p$ can be also removed, as the vertices belong to only one of the total $P$ contours.

Then, the speed of $\vec{v}_i$ is discretized using the artificial time-step parameter $\delta$, as the displacement
  $\frac{\partial \vec{v}_i}{\partial t} = \vec{v}_i(\delta = t+1) - \vec{v}_i(\delta = t)$:

  \begin{equation}
  \frac{\partial}{\partial t} \int_\Omega F(\vec{r}) \, d\vec{r} =
  - \underset{i}{\sum} w_{p,i} F(\vec{v}_i) \frac{\partial \vec{v}_i}{\partial t} \cdot \hat{\vec{n}}_i.
  \label{eq:regseg-shape_gradient_disc1}
  \end{equation}

Since the energy functional is defined over competing regions, the displacement of $\vec{v}_i$ will cause
  an energy exchange between the limiting regions, and therefore $F(\vec{r})$ must be split in
  two terms, $F_{in}(\vec{r})$ corresponding to the interior region and $F_{out}(\vec{r})$ to the exterior:

  \begin{equation}
  \frac{\partial}{\partial t} \int_\Omega F(\vec{r}) \, d\vec{r} =
  - \underset{i}{\sum} \, \frac{\partial \vec{v}_i}{\partial t} \cdot
  \underbracket{w_{p,i} \, \Big[ F_{out}(\vec{v}_i) - F_{in}(\vec{v}_i) \Big] \hat{\vec{n}}_i}_{\bar{s}_i \text{ in Figure 1}}.
  \label{eq:regseg-shape_gradient_disc2}
  \end{equation}}

\revcomment[R\#3-C.22]{%
Then, we identify the shape gradient contribution $\vec{g}_k$ on the coefficients $\vec{u}_k$ of the B-spline grid:

  \begin{equation}
  \label{eq:regseg-gradient_wshape}
  \begin{split}
  \vec{g}_k &= - \underset{i}{\sum} \left\langle \frac{\partial \vec{v}_i'}{\partial \vec{u}_k}, \bar{s}_i'\right\rangle \\
  \text{with }
  \bar{s}_i' &= w_i \left[ \mdist{f_i'}{out} - \mdist{f_i'}{in} \right] \, \hat{\vec{n}}_i, \\
  \text{and }
  \frac{\partial \vec{v}_i'}{\partial \vec{u}_k} &=
  \frac{\partial}{\partial \vec{u}_k} \left\{ \vec{v}_i + \sum_k \psi_k(\vec{v}_i) \vec{u}_k \right\} = \psi_k(\vec{v}_i)\, \hat{\vec{e}},
  \end{split}
  \end{equation}%
  where $\hat{\vec{e}}$ is the coordinates system's unit vector.
Therefore, the shape gradients projected to the grid of B-spline control points read:

\begin{equation}
  \vec{g}_k = - \underset{i}{\sum} \bar{s}_i \cdot \psi_k(\vec{v}_i) \, \hat{\vec{e}}.
  \label{eq:regseg-shape_gradient_final}
\end{equation}}

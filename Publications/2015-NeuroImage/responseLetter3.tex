\documentclass[9pt]{memoir}

\usepackage[sort&compress]{natbib}
\usepackage{xcolor}
\usepackage[T1]{fontenc}
\usepackage[utf8]{inputenc}
\usepackage{charter}

\usepackage{xcite}
\externalcitedocument{00-main}

\usepackage[colorlinks=true]{hyperref}
\usepackage{doi}
\usepackage{nameref}

\usepackage{amsmath}
\usepackage{amssymb}

\newcounter{reviewpoint}
\makeatletter
\newenvironment{reviewintro}%
{\begingroup%
 \color{black!60}
 \fontshape{it}\selectfont %
}
{\endgroup\par\medskip}


\newenvironment{reviewpoint}%
{\refstepcounter{reviewpoint}\par\medskip\vspace{3ex}\hrule\vspace{1.5ex}\par\noindent%
   {\fontseries{b}\selectfont Comment \arabic{reviewpoint}:}
   \begingroup%
   \color{black!60}
   \fontshape{it}\selectfont %

}
{\endgroup\label{com:\thereviewpoint}\par\medskip}
\def\reviewpointautorefname{Comment}
\makeatother


\newcommand{\reply}{\par\fontshape{n}\selectfont \noindent \textbf{Reply}:\ }

\setlrmarginsandblock{3cm}{2.5cm}{*}
\setulmarginsandblock{2.5cm}{2.5cm}{*}
\checkandfixthelayout

\begin{document}
\hypersetup{linkcolor=black!60, citecolor=black!60, urlcolor=black!60}

\section*{Response letter to the manuscript no. NIMG-15-2877R1}

\bigskip
\noindent We thank the editors and the reviewers for revising our manuscript no. NIMG-15-2877R1 submitted to NeuroImage.
We appreciate the comments made by both reviewers, and considered them fully to improve our paper.
Please, find in the following paragraphs our point-by-point answer to the concerns raised by the reviewers,
  along with the description of the corresponding changes in the manuscript.
Text modifications in the manuscript are highlighted with a new font color, and attached notes indicate 
  which reviewer and comment the corresponding change is addressing to.

\bigskip
\bigskip
\subsection*{Reviewer \#1:}
\begin{reviewintro}
In general, I think the authors have done a good job of restructuring the paper to qualify the goals and implementation of the method. I have a few points.
\end{reviewintro}
\begin{reviewpoint}
The authors suggest that diffusion tractography is typically performed by mapping segmentations to the data. This is becoming less true. The HCP for one performs whole brain connectivity analysis. Nevertheless, these frameworks still require registration to structural space for projection to surface mesh representations and/or coaligning data. This may be worth mentioning or clarifying.
\end{reviewpoint}
\reply{%
We have introduced this discussion in lines 351--355.
}
\begin{reviewpoint}
I still find figure 2 really difficult to read anything from. It's not easy to read the labels in the coloured text, nor see the different histograms in figure 2 A. This is partly because many of the colours are light very similar i.e. for WM and bst. It does occur to me that not all distributions change that greatly with registration. Maybe stripping back figure A to only include WM, GM and possibly VdGM might help?
\end{reviewpoint}
\reply{%
We have made the following editions in this figure to make it more clear:
\begin{itemize}
\item Changed color palette to the ``Set1'' palette of the ColorBrewer tool\footnote{\url{http://colorbrewer2.org/}}, which has
  darker colors.
\item The darker colors allowed for thiner lines to represent the contours of the joint density plot in panel A, improving
  readability.
\item Better positioned labels in the panel A of this figure, using the Fruchterman-Reingold force-directed algorithm.
\item Matched the fonts of axis labels with the font used in the NeuroImage journal.
\end{itemize}
}
\begin{reviewpoint}
I find the clarification of how tratography might be unwarped in the response very helpful, but some clarification of the manuscript might be helpful. What do the authors mean by ``the resampling of the DWIs introduced by the correction method can be avoided, either by performing the posterior processing on the native dMRI space''. How does the native dMRI space differ from the DWIs here, or what is meant by posterior processing.
\end{reviewpoint}
\reply{%
We have reformulated that particular sentence (lines 57--59) and added a corresponding discussion in lines 348--356.
These editions are highly related to Comments \ref{com:1}, \ref{com:5} and \ref{com:6}.
}
\begin{reviewpoint}
In the appendix I think the B-spline variable (Eqs A.7 A.8)  should be defined in the text again
\end{reviewpoint}
\reply{%
We have defined introduced the definition of $\mathbf{g}_k$ in lines 122--123, right before the
  reference to the mentioned equations.
}


% ------------------------------------------------------------------- Reviewer 3
\bigskip
\bigskip
\subsection*{Reviewer \#3:}

\begin{reviewintro}
This manuscript is much improved, in my opinion. However, there are some high-level clarifications that are present in the author response letter, but that are not clear to the reader in the manuscript. I also noticed a few small things to fix. My suggestions follow.
\end{reviewintro}

\begin{reviewpoint}
Please clarify that the authors do not consider this to be an EPI distortion correction method, but rather a surface alignment method or similar. This was not clear to me from reading the paper the first time. I assumed EPI distortion correction was a primary goal, but it seems rather that the primary goal is to accurately segment the distorted diffusion images. In fact, the distortions are not corrected in the EPI data.
\end{reviewpoint}
\reply{%
We have explicitly stated this in a new sentence (lines 348--349).
}
\begin{reviewpoint}
Please discuss the fact that when this method is applied, the tractography is done using data that have not been corrected for EPI distortions.
\end{reviewpoint}
\reply{%
Right after the previous edition, in lines 349--350 we introduce the proposal of this methodological variation.
The discussion and proposal of a future research line is then done in lines 355--356, after some further details added
  to address the \autoref{com:1}.
}
\begin{reviewpoint}
For the reader, the overall comparison between the two methods is confusing in the introduction, where it is proposed to register surfaces to diffusion image space, yet the comparison is b0 to T2. A more appropriate way to present this comparison is t2 to b0, so that the result is also in the diffusion image space, if this is what was done. This is in fact a common processing step, so I assume this is how the experiments were performed. Please clarify.
\end{reviewpoint}
\reply{%
The reviewer is right, we may have oversimplified the details about the 
  \emph{b0}-to-T2w registration.
We have added the text in lines 227--232 to describe how the registration works.
The direction of the transform is \emph{b0}-to-T2w (as implemented in \emph{ExploreDTI})
  and therefore, the result is a mathematical transformation which maps the coordinates
  of the T2w image to coordinates in the \emph{b0}.
This way, it is possible to obtain the corresponding intensity of the \emph{b0} and
  resample it in the reference space of the T2w.
Since the surfaces are defined by their vertices in reference space, the transform
  obtained in the \emph{b0}-to-T2w direction of mapping can be used to warp the
  locations of the vertices into \emph{b0} space.
Therefore, we have the surfaces on reference (undistorted) space, and we warp them
  into diffusion space using three transforms: 1) the ground truth given by the realistic
  fieldmaps, 2) \emph{regseg} and 3) the \emph{b0}-to-T2w registration method.
The error computations are done in diffusion space, 2) and 3) against 1).
This detail of which coordinate system supports the error measurement is included
  in the new version in lines 238--239.
}
\begin{reviewpoint}
In Figure 6 A, is this output coordinate system the diffusion coordinate system as expected from the description of the method? So then the output in T2B is T2 to b0?
\end{reviewpoint}
\reply{%
As described in \autoref{com:7}, yes, the coordinate system in which the surfaces are
  compared is the diffusion space.
Generally, when we speak of \emph{b0}-to-T2w registration, we identify the direction of
  the mapping with the displacement of the information of the \emph{b0} image into the
  T2w-image grid.
However, the transformation underlying works in the opposite way: for each point in
  the reference space computes the corresponding point in the moving image.
The objects defined explicitly, like the surfaces, in the reference space can be directly
  mapped to moving space using this transform.
This gives the impression that the transform goes in the opposite direction, and we
  understand that this is potentially what was not completely clear in the former
  version of the manuscript.
}
\begin{reviewpoint}
The ``undistorted'' HCP data is not fully defined in the ``Real datasets'' section. Because this undistortion method serves as the ground truth for the presented experiments, it should be explained briefly how the undistortion was done for the HCP data. One sentence is sufficient, and will allow the reader better understand the results of this paper.
\end{reviewpoint}
\reply{%
We have introduced this description in lines 183--187, along with including the detail of the actual HCP release of the data used in l. 177.
}
\begin{reviewpoint}
Minor points:

The graphical abstract says subpixel accuracy. I think you mean subvoxel (there are no pixels really in 3D MRI images).

``preliminar DTI'' -> preliminary
\end{reviewpoint}
\reply{%
We have updated the graphical abstract with the term ``subvoxel''.
We have replaced the six appearances of the term ``pixel'' by ``voxel'', including the ``subpixel'' present in line 61.
We also have fixed the misspelling of ``preliminar'' of line 339.
}

%\bibliographystyle{mystyle}
%\bibliography{Remote}

\end{document}

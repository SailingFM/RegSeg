% -*- root: 00-main.tex -*-
Current applications of whole-brain tractography on \acrlong*{dmri} data require highly precise
  delineations of anatomical structures, usually projected from anatomical \acrlong*{t1} images
  through registration.
We propose \regseg{}, a simultaneous segmentation and registration method
  that uses active contours without edges extracted from structural images.
The contours evolve through a free-form deformation field supported by B-Spline basis
  optimally mapping the contours onto the data in the target space.
We tested the functionality of \regseg{} on four digital phantoms warped with known and
  randomly generated deformations, to conclude that the accuracy achieved is subvoxel.
We then apply \regseg{} to perform registration/segmentation task on 16 real diffusion MRI
  datasets from the \acrlong*{hcp}, warped with realistic and nonlinear distortions typically
  present in these data.
We compute the misregistration error of the contours estimated by \regseg{} with respect
  to their theoretical location using the ground-truth, obtaining a 95\% CI of 0.56 - 0.66 (mm)
  distance between corresponding mesh vertices, below the 1.25mm isotropic resolution of the images.
We also compared the performance to a widely-used registration tool, to find out
  that \regseg{} overperformed the alternate method in our settings.

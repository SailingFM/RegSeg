% -*- root: 00-main.tex -*-
Current methods for processing \gls*{dmri} to map the connectivity of the human brain
  require precise delineations of anatomical structures.
This requirement has been approached \revcomment[R\#3-C.33]{by} either segmenting the data in
  native \gls*{dmri} space or mapping the structural information from \gls*{t1} images.
The characteristic features of diffusion data in terms of \acrlong*{snr}, resolution, as well
  as the geometrical distortions caused by the inhomogeneity of magnetic susceptibility
  across tissues hinder both solutions.
Unifying the two approaches, we propose \regseg{}, a surface-to-volume nonlinear
  registration method that segments homogeneous regions within multivariate images by mapping
  a set of nested reference-surfaces.
Accurate surfaces are extracted from a \gls*{t1} image of the subject, using as target image
  the bivariate volume comprehending the \gls*{fa} and the \gls*{adc} maps derived from the
  \gls*{dmri} dataset.
We first verify the accuracy of \regseg{} on a general context using digital phantoms.
Then we establish an evaluation framework using undistorted \gls*{dmri} data from the \gls*{hcp}
  and known deformations derived from real inhomogeneity fieldmaps.
We analyze the performance of \regseg{} computing the misregistration error of the surfaces estimated
  after being mapped with \regseg{} onto 16 datasets from the \gls*{hcp}.
The distribution of errors shows a 95\% CI of 0.56--0.66 mm, that is below the \gls*{dmri}
  resolution (1.25 mm, isotropic).
Finally, we cross-compare the proposed tool against a nonlinear \lowb{}-to-T2w registration
  method, thereby obtaining a significantly lower misregistration error with \regseg{}.
Therefore, we demonstrate that \regseg{} allows the accurate mapping of structural information
  in \gls*{dmri} space, enabling the application of new structure-informed techniques in
  the connectome extraction.
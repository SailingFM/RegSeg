% -*- root: 00-main.tex -*-
Current applications of whole-brain tractography to \acrlong*{dmri} data require highly precise
  delineations of anatomical structures, which are usually projected from \acrlong*{t1} images
  by registration.
In this study, we propose \regseg{}, which is a simultaneous segmentation and registration method
  that uses active contours without edges extracted from structural images.
The contours evolve through a free-form deformation field supported by the B-spline basis
  to optimally map the contours onto the data in the target space.
We tested the functionality of \regseg{} using four digital phantoms warped with known and
  randomly generated deformations, where subvoxel accuracy was achieved.
We then applied \regseg{} to a registration/segmentation task using 16 real diffusion MRI
  datasets from the \acrlong*{hcp}, which were warped by realistic and nonlinear distortions that are typically
  present in these data.
We computed the misregistration error of the contours estimated by \regseg{} with respect
  to their theoretical location using the ground truth, thereby obtaining a 95\% CI of 0.56--0.66 mm
  distance between corresponding mesh vertices, which was below the 1.25 mm isotropic resolution of the images.
We also compared the performance of our proposed method with a widely used registration tool, which showed
  that \regseg{} outperformed this method in our settings.
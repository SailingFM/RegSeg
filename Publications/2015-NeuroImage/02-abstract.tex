% -*- root: 00-main.tex -*-
\newcomment[RV\#1(C.1)]{%
Current methods using diffusion MRI to map the microstructure and connectivity of the
  human brain increasingly require precise delineations of anatomical structures.
The problem is typically solved by projecting the structural information extracted
  in T1-weighted images, by means of nonlinear registration since diffusion
  MRI data present geometrical distortions.
Here we propose \regseg{}, which segments homogeneous regions within multivariate
  images by projecting a set of nested surfaces extracted from structural images.}
The surfaces evolve through a free-form deformation field supported by the B-spline basis
  to optimally map the contours onto the data in the target space.
We tested the functionality of \regseg{} using four digital phantoms warped with known and
  randomly generated deformations, where subvoxel accuracy was achieved.
\newcomment[RV\#2(C.11)]{%
We proposed a set of surfaces that define a segmentation model of the fractional anisotropy and
  the apparent diffusion coefficient maps derived from diffusion MRI datasets.}
These surfaces extracted from the same-subject T1-weighted image successfully segmented 16 real
  diffusion MRI datasets from the Human Connectome Project that were warped by realistic and nonlinear
  distortions.
We computed the misregistration error of the surfaces estimated by \regseg{} with respect
  to their theoretical location using the ground truth, thereby obtaining a 95\% CI of 0.56--0.66 mm
  distance between corresponding mesh vertices, which was below the 1.25 mm isotropic resolution of the images.
We also compared the performance of the proposed method with a widely used registration tool, which showed
  that \regseg{} outperformed this method in our settings.
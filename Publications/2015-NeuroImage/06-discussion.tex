% -*- root: 00-main.tex -*-
\section{Discussion}
\label{sec:regseg-discussion}
\revcomment[C10 (R.2)]{%
We present \regseg{}, a simultaneous segmentation and registration method that
  maps a set of nested surfaces into a multivariate target-image.}
\revcomment[C1 (R.1)]{%
The nonlinear registration process evolves driven by the fitness of the
  picewise-smooth classification of voxels in the target volume imposed
  by the current mapping of the surfaces.}
\revcomment[moved]{%
We propose \regseg{} to map anatomical information extracted from \gls*{t1}
  images into the corresponding \gls*{dmri} of the same subject.
Previously, joint segmentation and registration has been applied successfully to
  other problems such as longitudinal object tracking \citep{paragios_level_2003}
  and atlas-based segmentation \citep{gorthi_active_2011}.
The most common approach involves optimizing a deformation model (registration)
  that supports the evolution of the active contours (segmentation),
  like \cite{paragios_level_2003,yezzi_variational_2003}.}
% Conversely, in structure-informed segmentation, the sources of variability are the geometrical distortions
%   after imaging and the anatomical evolution in longitudinal studies.}
\revcomment[C3 (R.1)]{%
\Regseg{} can be seen as a particular case of atlas-based segmentation-registration methods,
  replacing the atlas by the structural image of the subject (\emph{structure-informed segmentation}).
The main difference of atlas-based segmentation and the application at hand is the resolution of the
  target image.
Atlas-based segmentation is typically applied on high-resolution images.
A comprehensive review of joint segmentation and registration methods applied in atlas-based
  segmentation is found in \citep{gorthi_active_2011}.
They also propose a multiphase level-set function derived from a labeled atlas to implement the
  active contours that drive the atlas registration.
Alternatively, \regseg{} implements the active contours with a hierarchical set of explicit
  surfaces (triangular meshes) instead of the multiphase level sets, and registration
  is driven by shape-gradients \citep{herbulot_segmentation_2006}.
The use of explicit surfaces enables segmenting \gls*{dmri} images below pixel resolution.}

\revcomment[C3 (R.1)]{%
An important antecedent of \regseg{} is \emph{bbregister} \citep{greve_accurate_2009}.
The tool has been widely adopted as the standard registration method to be used along with the \gls*{epi}
  correction of choice.
It implements a linear mapping and uses 3D active contours \emph{with edges} to
  search for intensity boundaries in the \lowb{} image.
The active contours are initialized using surfaces extracted from the \gls*{t1} using
  \emph{FreeSurfer} \citep{fischl_freesurfer_2012}.
To overcome the problem of nonlinear distortions, \emph{bbregister} excludes from the
  boundary search those regions that are typically warped.
Indeed, the distortion must be addressed separately because it is not supported by
  the affine transformation model.}
\revcomment[C22 (R.2)]{%
Conversely, the deformation model of \regseg{} is nonlinear and the active contours are
  \emph{without edges} \citep{chan_active_2001} since the \gls*{fa} and \gls*{adc} maps
  do not present steep image gradients (edges) but the anatomy can be identified
  by looking for piece-wise smooth homogeneous regions.}

\revcomment[C3 (R.1)]{%
Recently, \cite{guyader_combined_2011} proposed a simultaneous segmentation and
  registration method in 2D using level sets and a nonlinear elasticity smoother on the
  displacement vector field, which preserves the topology even with very large deformations.
\Regseg{} includes an anisotropic regularizer for the displacements field proposed by
  \cite{nagel_investigation_1986}.
This regularization strategy conceptually falls in the midway between the Gaussian smoothing
  generally included in most of the existing methodologies, and the complexity of
  the elasticity smoother of \cite{guyader_combined_2011}.}
\revcomment[C3 (R.1)]{%
Other minor features that differ from current methods in joint segmentation and registration are
  the support of multivariate target-images and the efficient computation of the shape-gradients
  implemented with sparse matrices.}

We verified that precise segmentation and registration of a set of surfaces into multivariate
  data is possible on digital phantoms.
We randomly deformed four different phantom models to mimic three homogeneous regions
  (\gls*{wm}, \gls*{gm}, and \acrlong*{csf}) and we used them to simulate \gls*{t1}
  and \gls*{t2} images at two resolution levels.
We measured the Hausdorff distance between the contours projected using the
  ground-truth warping and the estimations found with \regseg{}.
We concluded that the errors were significantly lower than the voxel size.
We also assessed the 95\% \gls*{ci}, which yielded an aggregate interval of
  0.64--0.66 [mm] for the low resolution phantoms (2.0 mm isotropic voxel) and
  0.34--0.38 [mm] for the high resolution phantoms (1.0 mm isotropic).
Therefore, the error was bounded above by half of the voxel size.
\revcomment[C8 (R.1)]{%}
The distributions of errors along surfaces varied importantly depending on the shape of the
  phantom (see \autoref{fig:regseg-phantom}B).
The misregistration error of the ``gyrus'' phantom showed a much lower spread than that
  for the other shapes.
We argue that the symmetry of those other shapes posed difficulties in driving the contours
  towards the appropriate region due to \emph{sliding} displacements between the
  surfaces and their ground-truth position.
The effect is not detectable by the active contours framework, but it is controllable
  increasing the regularization constraints.
When \regseg{} is applied on real datasets, this surface sliding is negligible for the
  convoluted nature of cortical surfaces and the directional restriction of the
  distortion.}

We evaluated \regseg{} in a real environment using the experimental framework presented
  in \autoref{fig:regseg-evworkflows}.
We processed 16 subjects from the \gls*{hcp} database using both \regseg{}
  and an in-house replication of the \acrfull*{t2b} method.
\Regseg{} obtained very high accuracy, with an aggregate 95\% \gls*{ci} of 0.56--0.66 [mm], which was
  below the pixel size of 1.25 mm.
The misregistration error that remained after \regseg{} was significantly lower ($p <$ 0.01) than the
  error corresponding to the \gls*{t2b} correction according to Kruskal-Wallis H-tests
  (\autoref{tab:results_real}).
Visual inspections of all the results \citepalias[section S5]{esteban_useful_2015} and the violin plots in
  \autoref{fig:regseg-results_real} confirmed that \regseg{} achieved higher accuracy
  than the \gls*{t2b} method in our settings.
We carefully configured the \gls*{t2b} method using the same algorithm and the
  same settings employed in a widely-used tool for \gls*{dmri} processing.
However, cross-comparison experiments are prone to the so-called \emph{instrumentation bias}
  \citep{tustison_instrumentation_2013}.
Therefore, these results did not prove that \regseg{} \emph{is better than} \gls*{t2b},
  but indicated that \regseg{} is a reliable option in this application field.
\revcomment[C5 (R.1)]{%
Finally, we also contributed proposing a piecewise-smooth segmentation model defined by
  a selection of nested surfaces to partition the multispectral space
  comprehending the \gls*{fa} and the \gls*{adc} maps and ultimately identify these
  structures in \gls*{dmri} space.
We also demonstrated the smoothness of the objective function on five of the real datasets
  \citepalias[figure S2]{esteban_useful_2015}, taking advantage of the directional
  restriction of possible distortions.
However, convergence of \regseg{} can be severely affected when surfaces are not dense enough.
Using the digital phantoms, we decimated the surfaces by large factors.
Severely decimated surfaces introduced a bias that displaced the zero of the gradients from
  the minimum of the objective function.
Thus, the surfaces must be densely sampled to avoid the shape-gradients drive the algorithm
  in a wrong direction impeding convergence.
}%

\revcomment[C9 (R.2)]{%
The proposed application of the method in the task of identifying structural information
  in \gls*{dmri} images is an active field of research \citep{jeurissen_tissuetype_2015}.
Current processing steps involved in the connectome extraction require a precise segmentation
  of anatomical structures in the diffusion space.
Some examples of these processing tasks are the structure-informed reconstruction of \gls*{dmri}
  data \citep{jeurissen_multitissue_2014,daducci_accelerated_2015}, the anatomically constrained
  tractography \citep{smith_anatomicallyconstrained_2012}, and the imposition of the cortical
  parcellation mapped from the \gls*{t1} image \citep{hagmann_mapping_2008}.
The problem was firstly addressed using image segmentation approaches in the native diffusion
  space, without definite and compelling results.}
\revcomment[C18 (R.2)]{%
With the introduction of retrospective correction methods for the \emph{\gls*{epi} distortions}
  and image registration approaches, the task has been typically solved in a two-step approach.
First, the \glspl*{dwi} are corrected for \emph{\gls*{epi} distortions} by estimating
  the nonlinear-deformation field from extra MR acquisitions
  \citep{jezzard_correction_1995,chiou_simple_2000,cordes_geometric_2000,kybic_unwarping_2000}.
Second, mapping the structural information from the corresponding \gls*{t1} image
  using a linear registration tool like \emph{bbregister} \citep{greve_accurate_2009}.
The current activity on improving correction methods \citep{irfanoglu_drbuddi_2015} and
  the comeback of segmentation of \gls*{dmri} in its native space
  \citep{jeurissen_tissuetype_2015} proof the open interest of this application.
\Regseg{} addresses this joint problem in a single step and it does not require any additional
  acquisition other than the minimal protocol comprehending only \gls*{t1} and \gls*{dmri} images.
This situation is found commonly in historical datasets.
Moreover, since the structural information is projected into the native space of \gls*{dmri},
  the whole connectome extraction process can be performed without resampling data to an
  undistorted space.}

\revcomment[C9 (R.2)]{%
Beyond the presented application on \gls*{dmri} data, \regseg{} can be indicated in situations
  where there are precise surfaces delineating the structure, a target multivariate
  image in which the surfaces must be fitted, and the mapping between the surfaces and
  the volume encodes relevant physiological information, such as the normal/abnormal
  development or the macroscopic dynamics of organs and tissues.
For instance, \regseg{} may be applied in fields like neonatal brain image segmentation
  in longitudinal MRI studies of the early developmental patterns \citep{shi_neonatal_2010}.
In these studies, the surfaces obtained in a mature timepoint of the brain are retrospectively
  propagated to the initial timepoints, regardless the changes of the contrast and spatial
  development between them.
More generally, \regseg{} may also be applied to the personalized study of longitudinal alteration
  of the brain using multispectral images, for instance in the case of traumatic brain
  injury \citep{irimia_structural_2014} or in the monitorization of brain tumors
  \citep{weizman_semiautomatic_2014}.}
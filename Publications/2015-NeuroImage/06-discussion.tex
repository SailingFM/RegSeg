% -*- root: 00-main.tex -*-
\section{Discussion}
\label{sec:discussion}

\paragraph*{Accuracy tests}
The hypothesis underpinning our research was that image registration can be reliably performed
  by searching for homogeneous regions in the target image that correspond to precise contours
  from an atlas, or extracted from other image (i.e. a different time step).
We proved that active contours without edges can be used to successfully drive a
  deformation supported by B-Spline basis functions using digital phantoms.
We randomly deformed 4 different phantom models mimicking three homogeneous regions,
  and use them to simulate \gls*{t1} and \gls*{t2} images at two resolution levels.
After registration with \emph{regseg} we measured the Hausdorff distance between the
  projected contours using the ground-truth warping and our estimation.
We concluded that errors are significantly lower than voxel resolution.
We also assessed the 95\% \gls*{ci}, yielding an aggregate interval of
  0.64 - 0.66 [mm] in the low resolution phantoms (2.0mm isotropic voxel) and
  0.34 - 0.38 [mm] in the high resolution phantoms (1.0mm isotropic).
Therefore, we also conclude that the error is bounded above by the half of the
  voxel spacing.

\paragraph*{Application on real data}
We designed \emph{regseg} as a method to segment \gls*{dmri} data exploiting the
  anatomical information extracted in the corresponding \gls*{t1} image of the subject.
Applications of whole-brain tractography \citep{smith_tractbased_2006,craddock_imaging_2013}
  usually solve the problem in a two-step approach.
First, images are corrected for nonlinear distortions using auxiliary acquisitions
  such as fieldmaps \citep{jezzard_correction_1995}, \gls*{t2} images \citep{kybic_unwarping_2000},
  etc.
Second, the segmentation is projected from a reference \gls*{t1} image using linear
  registration \citep{greve_accurate_2009}.
\emph{Regseg} solves this joint problem in one single step, and it does not require any additional
  acquisition over the minimal protocol including \gls*{t1} and \gls*{dmri} images only.
This situation is commonly found in historical datasets.
Beyond the proposed application in \gls*{dmri} data, other potential uses of \emph{regseg} are
  atlas-based segmentation and tracking objects in time-series.

We evaluated \emph{regseg} in a real environment, using the experimental framework presented
  in \autoref{fig:evworkflows}.
We processed 16 subjects from the \gls*{hcp} database with both \emph{regseg}
  and an in-house replication of the \gls*{t2b} correction method.
\emph{Regseg} showed very high accuracy, with an aggregate 95\% \gls*{ci} of 0.56-0.66 [mm],
  bellow the pixel size of 1.25mm.
The misregistration error remaining after \emph{regseg} was significantly lower ($p < 0.01$) than the
  error corresponding to the \gls*{t2b} correction, using Kruskal-Wallis H-tests
  (\autoref{tab:results_real}).
Visual inspection of all the results (\suppl{section S5}) and the violin plots included in
  \autoref{fig:results_real} confirmed that \emph{regseg} overperformed the \gls*{t2b} method
  in our settings.
Even though we carefully configured the \gls*{t2b} method by using the same algorithm and
  same settings of a widely used software, cross-comparison experiments are prone to
  the so-called \emph{instrumentation bias} \citep{tustison_instrumentation_2013}.
Therefore, these results do not prove that the \emph{regseg} software \emph{is better than} the
  \gls*{t2b} method.
The results reported let us propose \emph{regseg} as a solid option in the application field.
Additionally, the \gls*{t2b} may introduce an additional (and small) error in the necessary
  registration of \gls*{t2} to \gls*{t1} space.

% \paragraph*{Prospects}
% First extensions of this work will study more appropriate features to build the energy functional
%   on, enabling to perform \emph{regseg} directly on the raw \gls*{dmri} data.
% A second outlook covers incorporating knowledge about the distortion by initializing our method
%   with the theory-based displacement field that can be estimated with fieldmaps.
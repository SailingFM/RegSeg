% @Author: Oscar Esteban
% @Date:   2015-02-20 15:40:09
% @Last Modified by:   Oscar Esteban
% @Last Modified time: 2015-02-23 13:43:56

\documentclass[a4paper]{report}

\usepackage{titlesec}
\newcommand{\sectionbreak}{\clearpage}

\usepackage{aliascnt}
\usepackage{relsize}
\usepackage[procnames]{listings}

\usepackage{natbib}
%\usepackage[style=numeric-comp,doi=false]{biblatex}

% *** GRAPHICS RELATED PACKAGES ***
\usepackage[table]{xcolor}
\usepackage{graphicx}
\usepackage{ifpdf}
\graphicspath{{figures/},{../Figures/}}

% *** MATH PACKAGES ***
\usepackage[cmex10]{amsmath}
\usepackage{amssymb}

\usepackage[T1]{fontenc}
\usepackage{charter}
%\usepackage[expert]{mathdesign}
%\usepackage[libertine,cmintegrals,cmbraces,vvarbb]{newtxmath}

% *** SPECIALIZED LIST PACKAGES ***
%\usepackage{algorithmic}
%\usepackage{algorithm}


% *** ALIGNMENT PACKAGES ***
%\usepackage{array}

% *** SUBFIGURE PACKAGES ***
\usepackage[font={small}]{caption}
%\usepackage{sidecap}

% *** FLOAT PACKAGES ***
\usepackage[framemethod=tikz]{mdframed}
%\usepackage{float}
% \usepackage{fixltx2e}
% \usepackage{stfloats}
% \usepackage{dblfloatfix}


% *** MISC UTILITY PACKAGES ***
%\usepackage{fancyhdr}
\usepackage{multicol}
\usepackage{tabularx}
\usepackage{longtable}
\usepackage{booktabs} % to use \toprule
\usepackage{makeidx}
%\usepackage[scale=2.0]{ccicons}
\usepackage[colorinlistoftodos]{todonotes}
\usepackage[marginparwidth=1.2in]{geometry}
\usepackage{changepage}
\usepackage{epigraph}
\usepackage[toc,nomain,acronym,shortcuts,translate=false]{glossaries}
\usepackage[hyphens]{url}
\usepackage[breaklinks,hidelinks]{hyperref}
\usepackage[hyphenbreaks]{breakurl} %fixes boxes spanning through pages

\renewcommand\thesection{S\arabic{section}}
\renewcommand\thesubsection{S\arabic{section}.\arabic{subsection}}

% Listings style -----------------------------------------------------------------------------
\definecolor{pink}{RGB}{255,0,90}
\definecolor{comments}{RGB}{50,120,110}
\definecolor{string}{RGB}{160,0,0}
\definecolor{keywords}{RGB}{0,150,0}
\def\listingsfont{\ttfamily}
\def\listingsfontinline{\ttfamily}

\lstnewenvironment{bashcode}[1][]
  {\lstset{language=bash}\lstset{%
  showstringspaces=false,
  formfeed=\newpage,
  tabsize=4,
  breaklines=true,
  basicstyle=\ttfamily\smaller\relax,
  keywordstyle=\color{keywords}\bfseries,
  commentstyle=\color{comments}\itshape,
  stringstyle=\color{string},
  showstringspaces=false,
  %identifierstyle=\color{green},
  procnamekeys={def,class},
  morekeywords={models, lambda, forms, as, from},
  numbers=none,
  frame=single,
  xleftmargin= 10pt,
  xrightmargin= 10pt,
  framexleftmargin=10pt,
  frameround=tttt,
  fillcolor=\color{gray!10},
  backgroundcolor=\color{gray!10}
}}
{}


\lstnewenvironment{pythoncode}[1][]
  {\lstset{language=bash}\lstset{%
  showstringspaces=false,
  formfeed=\newpage,
  tabsize=4,
  breaklines=true,
  basicstyle=\ttfamily\smaller\relax,
  keywordstyle=\color{keywords}\bfseries,
  commentstyle=\color{comments}\itshape,
  stringstyle=\color{string},
  showstringspaces=false,
  %identifierstyle=\color{green},
  procnamekeys={def,class},
  morekeywords={models, lambda, forms, as, from},
  numbers=left,
  numberstyle=\smaller\color{black!60},
  stepnumber=1,
  numbersep=5pt,
  frame=single,
  xleftmargin= 40pt,
  xrightmargin= 20pt,
  framexleftmargin=20pt,
  frameround=tttt,
  fillcolor=\color{gray!10},
  backgroundcolor=\color{gray!10}
}}
{}

\begin{document}
\title{Supplemental Materials: \emph{Simultaneous segmentation and registration of
diffusion MR images of the brain driven by active-contours}}
\author{Oscar Esteban}
\date{February 2015}

\maketitle

\section{Parameter settings and implementation details of \emph{regseg}}

\subsection{Implementation}
\paragraph{General} The \emph{regseg} registration and segmentation tool is written in C++, using ITK-4.6 as
  implementation core.
We designed a modular implementation, enabling multithreading in several pieces of the software,
  as the process is computationally expensive.
The tool generates a log-file in JSON format to easily inter-operate with secondary tools (such
  as the convergence report generation, \autoref{sec:convergence_evidence}).

\paragraph{Efficient interpolation using sparse matrices}
During the registration process, every iteration requires computing the product of all the speeds
  $\mathbf{s}'_i$ computed at the current position $\mathbf{v}'_i$ by the corresponding weights
  $\psi_k(\mathbf{v}_i)$ of interpolating functions (Eq. (15)).
As these weights can be computed once in the beginning of the process and they do not change along
  it, and they are different from zero only in the proximity of the corresponding control node $k$,
  they can be pre-cached in a sparse matrix.

\subsection{Interface and Settings}\label{sec:interface_settings}

\paragraph{Command-line interface}
The command line interface of \emph{regseg} supports general settings and level-wise settings.
For each multi-resolution level, its corresponding settings are added between brackets.

\begin{bashcode}
regseg -F fa.gz adc.nii.gz -P white.vtk pial.vtk -o myprefix [ -a 0.00000 -b 0.00000 --convergence-energy -t 1.0e-06 -w 60 --adaptative-descriptors --grid-spacing 16.0 -i 500 -s 0.001] [ -a 0. -b 0. --convergence-energy -t 1.e-08 -w 5 --grid-spacing 8.0 -i 250 -s 0.01]
\end{bashcode}


It is possible to get the description of available options running
  \lstinline!regseg -h!:

\begin{bashcode}
Usage:

General options:
  -h [ --help ]                         show help message
  -F [ --fixed-images ] arg             fixed image file
  -P [ --surface-priors ] arg           shape priors
  -T [ --surface-target ] arg           final shapes to evaluate metric (only
                                        testing purposes)
  -M [ --fixed-mask ] arg               fixed image mask
  -L [ --transform-levels ] arg         number of multi-resolution levels for
                                        the transform
  -o [ --output-prefix ] arg (=regseg)  prefix for output files
  -l [ --logfile ] arg                  log filename
  -v [ --monitoring-verbosity ] arg (=1)
                                        verbosity level of intermediate results
                                        monitoring ( 0 = no output; 5 = verbose
                                        )

Optimizer options (by levels):
  -a [ --alpha ] arg              alpha value in regularization
  -b [ --beta ] arg               beta value in regularization
  -s [ --step-size ] arg          step-size value in optimization
  -g [ --gradient-scales ] arg    alpha value in regularization
  -r [ --learning-rate ] arg      learning rate to update step size
  -i [ --iterations ] arg         number of iterations
  -w [ --convergence-window ] arg number of iterations of convergence window
  -t [ --convergence-thresh ] arg convergence value
  --grid-size arg                 size of control points grid
  --grid-spacing arg              spacing between control points
  -u [ --update-descriptors ] arg frequency (iterations) to update descriptors
                                  of regions (0=no update)
  --adaptative-descriptors        recomputes descriptors more often at the
                                  beginning of the process
  --convergence-energy            disables lazy convergence tracking: instead
                                  of fast computation of the mean norm of the
                                  displacement field, it computes the full
                                  energy functional

Functional options (by levels):
  --smoothing arg               apply isotropic smoothing filter on target
                                image, with kernel sigma=S mm.
  --smooth-auto                 apply isotropic smoothing filter on target
                                image, with automatic computation of kernel
                                sigma.
  --uniform-bg-membership       consider last ROI as background and do not
                                compute descriptors.
  -d [ --decile-threshold ] arg set (decile) threshold to consider a computed
                                gradient as outlier (ranges 0.0-0.5)

\end{bashcode}

\paragraph{\emph{Nipype} interface}
Our registration algorithm is released with a \emph{nipype Interface} packaged in
  \lstinline!pyacwereg.interfaces.acwereg!.
This interface has been comprehensively used in the evaluation workflows.

\begin{pythoncode}
from pyacwereg.interfaces.acwereg import ACWEReg
regseg = ACWEReg()
regseg.inputs.in_fixed = ['T1w.nii.gz', 'T2w.nii.gz']
regseg.inputs.in_pior = ['csf.vtk', 'white_lh.vtk', 'white_rh.vtk',
                         'pial_lh.vtk', 'pial_rh.vtk']
ifresult = regseg.run()
\end{pythoncode}



\subsection{Convergence evidencing}\label{sec:convergence_evidence}
In order to track the evolution of the registration process, several internal variables
  are saved in the JSON log-file.
Using the JSON log-file as input for the \emph{nipype Interface}
  \lstinline!pyacwereg.interfaces.ACWEReport!, it is straightforward to obtain
  a visual assessment document presenting the convergence.
Online checking is also possible as the algorithm writes to the standard output as well.

\begin{pythoncode}
from pyacwereg.interfaces.acwereg import ACWEReport
report = ACWEReport()
report.inputs.in_log = `myprefix.log'
ifresult = report.run()
\end{pythoncode}

\begin{figure*}[!ht]
	\includegraphics[width=\textwidth]{figures/Suppl-figure02.pdf}
	\caption{The evolution of the registration and segmentation process can be
	  checked using the \emph{Convergence report},
	  easily generated using the appropriate \emph{nipype Interface}.
	The report comprehends several plots tracking the evolution of the algorithm and several
	  features to help researchers tune up the algorithm in their application.}%
	  \label{fig:convreport}
\end{figure*}

\section{Instruments for evaluation}

\section{Model considerations}

\begin{figure*}[!ht]
	\includegraphics[width=\textwidth]{figures/Suppl-figure01.pdf}
	\caption{Evaluating the joint distribution}\label{fig:jointplot}
\end{figure*}

\section{Extended results on phantom data}

\section{Extended results on real data}

\end{document}
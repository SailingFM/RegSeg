\section{Data and experiments}
\label{sec:experiments}
%
\subsection{Shape prior}
%
As described in \autoref{sec:introduction}, the general situation in
the connectivity pipelines consists of having 
a reliable segmentation obtained from the high resolution \ac{t1} 
reference image. Therefore, a precise location of the tissue interfaces
of interest is available in a reference space. Given that the anatomical 
reference segmentation is beyond the scope of this manuscript, we simply 
rely on the shape priors obtained from the models.

%
\subsection{Synthetic gray-scale data}
%
The first, toy example is inspired by a problem shown for coupled CSF/GM 
and GM/WM segmentation in \cite{MacDonald2000}. \textbf{FIXME description of the model, expected difficulties (PVE -- lack of contour in sulcus), and how proposed model is expected to cope with this}.

%
\subsection{Simulated diffusion data}
%
In order to demonstrate the specific functionality of the proposed
methodology and characterize its possibilities with diffusion data, we developed a
synthetic model and simulated its corresponding \ac{dwi}
raw signal as described in \citep{tuch_q-ball_2004}. 

This second model consists of several spherical shapes emulating
the different brain tissues (see \autoref{fig:fa}, first row). 
We reconstructed the \ac{dwi} signal with standard procedures to 
approximate the environment to the real one at maximum. \\

The target \ac{dwi} data is characterized by its distortions and its
low resolution (typically around $2.2x2.2x3mm^3$). Depending on the
posterior reconstruction methodology and the angular resolution
intended, the \ac{dwi} raw data has to be processed in order to
extract the information in a manageable manner. The properties of
the reconstructed tensors and derived scalar maps have been
studied and presented on \cite{ennis_orthogonal_2006}. Based on their
findings, the proposed energy model adapts to the \ac{fa} and \ac{md}
for their properties.
Whereas \ac{fa} describes the \emph{shape} of diffusion, 
the \ac{md} depicts the \emph{magnitude} of the process. 
There exist two main reasons to justify their choice. 
First, they are well-understood and standardized in clinical routine.
Second, together they contain most of the information that is
usually extracted from the \ac{dwi}-derived scalar maps. \\

For this model, we created manually a sound distortion visually similar
to real \ac{epi} distortions. We interpolated the distortion to a 
dense deformation field, necessary for warping the raw \ac{dwi} simulated
data. Once the signal was deformed, we proceeded to reconstruct the
\ac{dti} and subsequently obtained the scalars of interest (\ac{fa}, \ac{md}).
Finally, we estimated their parameters using the tissue probability
distribution maps from the original model (\autoref{table:parameters}).

\begin{table}
\begin{tabular}{cccc}
         & $\mathbf{\mu}_{FA}$ & $\mathbf{\mu}_{MD}$ & $\mathbf{\Sigma}$ \\
\ac{wm}  & & & \\
\ac{gm}  & & & \\
\ac{csf} & & & \\
\end{tabular}
\caption{Model parameters}
\label{table:parameters}
\end{table}
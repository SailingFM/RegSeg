\section{Methods}
\label{sec:methods}
%
\subsection{Simulated datasets}
%
As described in \autoref{sec:introduction}, the general situation in
the connectivity pipelines consists on having 
a reliable segmentation obtained from the high resolution \ac{t1} 
reference image. Therefore, a precise location of the tissue interfaces
of interest is available in a reference space. Given that there is no 
interest on the anatomical reference segmentation,
we directly obtained the shape priors from the models. \\

On the other hand, the target \ac{dwi} data is characterized by its
low resolution (typically around $2.2x2.2x3mm^3$). Depending on the
posterior reconstruction methodology and the angular resolution
intended, the \ac{dwi} raw data has to be processed in order to
extract the information in a manageable manner. Particularly, we
will use the \ac{fa} and \ac{md} maps for convenience.
Whereas \ac{fa} describes the \emph{shape} of diffusion, 
the \ac{md} depicts the \emph{manitude} of the process. 
There exist two main reasons to justify their choice. 
First, they are well-understood and standardized in clinical routine.
Second, together they contain most of the information that is
usually extracted from the \ac{dwi}-derived scalar maps. \\

In order that demonstrating the functionality of the proposed
methodology and characterize its possibilities, we developed two
synthetic models and simulated their corresponding \ac{dwi}
raw signal as described in \citep{tuch_q-ball_2004}. 
(HERE WE NEED A GOOD DESCRIPTION OF THE DATA, directions, resolution, etc)
. The first model consists of several spherical shapes emulating
the different brain tissues (see \autoref{fig:fa}, first row). 
The second model is based on the BrainWeb dataset.
We reconstructed the \ac{dwi} signal with standard procedures to approximate the
environment to the real one at maximum. \\

\subsection{\acl{acwe}-like variational segmentation model}
%
Let us denote $\{c_i\}_{i=1..N_c}$ the nodes of a shape prior surface. In
our application, a precise \ac{wm}-\ac{gm} interface extracted from a
high-resolution reference volume. All the formulations can be naturally
extended to include more shape priors. On the other hand, we have a 
number of \ac{dwi}-derived features at each
voxel of the volume. Let us denote by $x$ the voxel and 
$f(x) = [ f_1, f_2, \ldots, f_N]^T(x)$ its associated feature vector.\\
%
The transformation from reference into \ac{dwi} coordinate space is 
achieved through a dense deformation field $u(x)$, such that:
%
\begin{equation}
c_i' = T\{c_i\} = c_i + u(c_i)
\end{equation}
% 
Since the nodes of the anatomical surfaces might lay off-grid, it is 
required to derive $u(x)$ from a discrete set of parameters $\{u_k\}_{k=1..K}$.
Densification is achieved through a set of associated basis functions 
$\Psi_k$ (e.g. rbf, interpolation splines):
%
\begin{equation}
u(x) = \sum_k \Psi_k(x) u_k
\end{equation}
%
Consequently, the transformation writes
%
\begin{equation}
\label{eq:transformation}
c_i' = T\{c_i\} = c_i + u(c_i) = c_i + \sum_k \Psi_k(c_i)u_k
\end{equation} 
%
% Comment: maybe this is not for IPMI 2013.
%Note that, since $c_i$ remains constant in the DW segmentation process,
%the values of $\Psi_k(c_i)$ can be precomputed. Also, provided compact 
%support of the basis functions, the system remains relatively sparse.\\
%
Based on the current estimate of the distortion $u$, we can compute 
``expected samples'' within the shape prior projected into the \ac{dwi}.
Thus, we now estimate region descriptors of the \ac{dwi} features 
$f(x)$ of the regions defined by the priors in \ac{dwi} space.
%
Using Gaussian distributions as region descriptors, we propose an
\ac{acwe}-like, piece-wise constant, variational image segmentation
model (where the unknown is the deformation field)
\cite{chan_active_2001}:
\begin{equation}
\label{eq:gaussian_energy}
E(u)= \sum_{\forall{R}} \int_{\Omega_R} (f-\mu_R)^T\Sigma_R^{-1}(f-\mu_R) dx
\end{equation}
where $R$ indexes the existing regions and the integral domains
depend on the deformation field $u$. Note
that minimizing this energy, $\argmin_u\{E\}$, yields the \ac{map} 
estimate of a piece-wise smooth image model affected by Gaussian 
additive noise. This inverse problem is ill-posed
\cite{bertero_ill-posed_1988,hadamard_sur_1902}.
In order to account for deformation field regularity and to render the 
problem well-posed, we include limiting and regularization terms into 
the energy functional \cite{morozov_linear_1975,tichonov_solution_1963}:
%
\begin{align}
\label{eq:complete_energy}
E(u) &= \sum_{\forall{R}} \lbrace \int_{\Omega_R} (f-\mu_R)^T\Sigma_R^{-1}(f-\mu_R) dx \rbrace \nonumber \\
&\quad + \alpha \int  \|u\|^2 dx + \beta \int \left( \|\nabla u_x\|^2 + \|\nabla u_y\|^2 + \|\nabla u_z\|^2\right) dx
\end{align}
%
These regularity terms ensure that the segmenting contours in 
\ac{dwi} space are still close to their native shape. The model
easily allows to incorporate inhomogeneous and anisotropic 
regularization \cite{nagel_investigation_1986} to better regularize
the \ac{epi} distortion. \\
%

At each iteration, we update the distortion along the steepest 
energy descent. This gradient descent step can be efficiently 
tackled by discretizing the time in a forward Euler scheme, 
and making the right hand side semi-implicit in the 
regularization terms:
%
\begin{align}
\frac{u^{t+1}-u^t}{\tau} &= - \sum_{i=1}^{N_c} \left[ \Delta E(f(c_i'))  \hat{n}_{c_i'} \Psi_{c_i}(x) \right] -\alpha u^{t+1} + \beta\Delta u^{t+1}
\end{align}
%
where the data terms remain functions of the current estimate 
$u^t$, thus $c_i' = c_i'(u^t)$. For simplicity on notation, we 
restricted the number of priors to only 1. We also defined 
$\Delta E(f(c_i')) = E_{out}(f(c_i')) - E_{in}(f(c_i'))$, 
and $E_R(f) = \sqrt{(f-\mu_R)^T\Sigma_R^{-1}(f-\mu_R)}$.
We applied a spectral approach to solve this implicit scheme:
%
\begin{equation}
u^{t+1} = \mathcal{F}^{-1}\left\{ \frac{\mathcal{F}\{u^t/\tau
- \sum_{i=1}^{N_c} \left[ \Delta E(f(c_i')) \hat{n}_{c_i'} \Psi_{c_i}(x) \right]  \}}{\mathcal{F}\{(1/\tau+\alpha)\mathcal{I}-\beta\Delta\}} \right\}
\end{equation}
%

\subsection{Experiment}
%
For both models, we created manually a sound distortion visually similar
to real \ac{epi} distortions. We interpolated the distortion to a 
dense deformation field, necessary for warping the raw \ac{dwi} simulated
data. Once the signal was deformed, we proceeded to reconstruct the
\ac{dti} and subsequently obtained the scalars of interest (\ac{fa}, \ac{md}) and estimated their parameters on the model.\\

We evaluate the performance of our methodology to estimate the deformation
field and compare it to the original synthetic deformation field applied to the data.

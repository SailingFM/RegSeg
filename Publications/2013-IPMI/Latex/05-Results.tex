\section{Results and discussion}
\label{sec:results}



\subsection{Synthetic gray-scale data}



\subsection{Simulated diffusion data}
%
The proposed method successfully reverted the synthetic distortion
field we applied to the data. With 16x16x16 control points, the
displacements field is dense enough to correctly represent the
synthetic field. (INCLUDE FIGURE). Figure XX shows the restored
image and a difference map with the original model (we can also
compute Dice indexes and that stuff).\\
%

\begin{figure}
\begin{tabular}{ccccc}
\includegraphics[width=0.19\textwidth]{fig_modelFA_00} &
\includegraphics[width=0.19\textwidth]{fig_modelFA_01} &
\includegraphics[width=0.19\textwidth]{fig_modelFA_02} &
\includegraphics[width=0.19\textwidth]{fig_modelFA_03} &
\includegraphics[width=0.19\textwidth]{fig_modelFA_04} \\
\includegraphics[width=0.19\textwidth]{fig_contours_00} &
\includegraphics[width=0.19\textwidth]{fig_contours_01} &
\includegraphics[width=0.19\textwidth]{fig_contours_02} &
\includegraphics[width=0.19\textwidth]{fig_contours_03} &
\includegraphics[width=0.19\textwidth]{fig_contours_04}
\end{tabular}
\caption{First row presents several slices along Z axis of the \ac{fa} map obtained 
after \ac{dti} fitting (using the original signal without simulated distortion) and
the corresponding \ac{wm}-\ac{gm} contour. Second row contains the warped \ac{fa}
map (using the distorted signal). The red contour corresponds to the interface
detected after registration. In blue color, the initial \ac{wm}-\ac{gm} contour
(the same as in previous row).}
\label{fig:fa}
\end{figure}
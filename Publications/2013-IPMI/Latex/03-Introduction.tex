\section{Introduction}
\label{sec:introduction}
%
\ac{dwi} is a widely used family of \ac{mr} techniques
\citep{sundgren_diffusion_2004} which recently has accounted for a growing
interest in its application to whole-brain structural connectivity analysis.
This emerging field, coined in 2005 as \emph{\ac{mr} Connectomics}
\citep{hagmann_diffusion_2005,sporns_human_2005}, currently includes a
large amount of imaging techniques for acquisition, processing, and analysis
specifically tuned for \ac{dwi} data.\\

The whole-brain connectivity analysis has arisen some challenges
that should be overcome in order to get reliable structural information
about the neuronal tracts from \ac{dwi} \cite{johansen-berg_using_2009,
jones_white_2012}. The earlier stages of these
processing pipelines generally include two necessary steps, brain tissue
segmentation on the diffusion space and the correction of geometrical
distortions inherent to the imaging techniques \citep{hagmann_mr_2012}.\\

In this work, we will refer as brain tissue segmentation to the precise
delineation of the \ac{csf}-\ac{gm} and \ac{gm}-\ac{wm} interface surfaces.
This segmentation is an important step on which strongly rely further
tasks. In tractography, a high-standard \ac{wm} mask is required. Otherwise,
there is an important risk for the algorithm to lose fiber bundles. This
requirement is usually solved in practice by plainly thresholding the 
\ac{fa} (a well-know scalar map derived from \ac{dwi} which depicts 
the isotropy of water diffusion inside the brain). 
Additionally, a precise location of the
\ac{gm}-\ac{wm} surface is required in the final steps to
achieve a consistent parcellisation of the cortex to represent the nodes 
of the output network. This parcellisation is generally defined in a 
high-resolution and better understood structural \ac{mri} of the same 
subject (e.g. T1 and/or T2 weighted acquisitions). Conversely, this 
problem is resolved with non-linear registration of a structural \ac{mri}
of the subject to the \ac{dwi} data. Even though some efforts have addressed
the study of the robustness of tractography versus the intra-subject variability
\cite{wakana_reproducibility_2007,heiervang_between_2006}, the results produced
are restricted to relevant regions of the brain. Therefore, extremely robust
and precise segmentation methods are required in the whole-brain application. \\

On the other hand, the \ac{dwi} data is usually obtained with \ac{epi}
acquisition techniques, that often suffer from severe distortions due to 
local field inhomogeneities. Generally, it is easily appreciated in the anterior
part of the brain, along the phase-encoded direction. Some methodologies have
been developed and generically named as \emph{\ac{epi}-unwarp} techniques
\cite{holland_efficient_2010,hsu_correction_2009,jezzard_characterization_2005,
reber_correction_2005}. They usually 
require the extra acquisition of the magnitude and phase of
the field (field-mapping), condition which is not always met. Some other 
methodologies do not make use the field-mapping, compensating the distortion
with non-linear registration from structural \ac{mri} or other means
\citep{andersson_modeling_2001}. To our knowledge, there exists no study
of the impact of the \ac{epi} distortion on the variability of tractography
results. \\

In this paper we propose a novel registration framework to simultaneously
solve the segmentation and distortion challenges, by exploiting as strong 
shape-prior the detailed anatomy extracted from anatomical \ac{mri}. This
is justified by the strong relationship between both problems, and the
advantage of a significant increase of coherence among these steps in
the complete processing pipeline.
We reformulate the segmentation problem as an inverse problem, where
we seek for an underlying deformation field (the distortion) mapping 
from the structural space into the diffusion space.
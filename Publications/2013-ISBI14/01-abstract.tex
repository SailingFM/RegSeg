\begin{abstract}
Connectivity analysis on diffusion MRI data of the whole-brain
suffers from distortions originated by the standard 
echo-planar imaging sequences used in acquisition. %
%These artifacts are present at interfaces between tissues,
%due to the inhomogeneity of the field derived from the 
% discontinuity of magnetic susceptibility.
These images show characteristic geometrical deformations
and signal destruction that suppose a significant pitfall
limiting the success of tractography algorithms.

Several retrospective correction techniques are readily
available. In this work, we use a digital phantom designed
for the evaluation of connectivity pipelines. We subject
the phantom to a ``theoretically correct'' and plausible
deformation that resembles the artifact under investigation.
We correct data back, with three standard methodologies
(namely fieldmap-based, reversed encoding-based, and
registration-based). Finally, we rank the methods based
on their geometrical accuracy, the dropout compensation,
and their impact on the resulting tractography.
\end{abstract}%
\begin{keywords}
susceptibility artifacts, diffusion MRI, tractography, connectivity.
\end{keywords}
\begin{abstract}

Connectivity analysis on \gls*{dmri} data of the whole-brain
suffers from distortions originated by the standard 
\gls*{epi} sequences used in acquisition. These artifacts
are present at interfaces between tissues, due to the inhomogeneity
of the field derived from the discontinuity of magnetic
susceptibility. As a result, registered signal shows characteristic
geometrical deformations and signal destruction. Thus, the artifact
is a significant pitfall that limits the success of tractography 
algorithms.

Several retrospective correction techniques are readily
available, generally inherited from \gls*{fmri} literature,
as those data are usually acquired with \gls*{epi} as well.
These correction methodologies have been evaluated and 
compared in physical phantoms designed for \gls*{fmri} 
experiments, but little is known about their impact
on \gls*{dmri} and more precisely on the resulting 
tractography.

In this work, we use a renowned digital phantom designed
for assessing tractography algorithms and subject it to
a theoretically correct and plausible deformation. We corrected
data with several available methodologies and provide
a ranking based on the accuracy of the resulting tractographies.

\end{abstract}

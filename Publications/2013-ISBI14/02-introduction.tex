\section{INTRODUCTION}

In-vivo whole-brain connectivity analysis has been a
research topic of high interest for the last
5 years. 
\Gls*{dmri} can be used to probe the
orientation of fiber bundles within the brain,
generally applying \gls*{epi} sequences.
After a signal reconstruction step, 
tractography algorithms draw a map of the sampled 
structures.
These maps can represent the actual trajectories
of fiber bundles (deterministic tractography) or
pixel-wise probability of connection to a certain origin
(probabilistic tractography). Finally, the
information about these connections is collected
into a network matrix that can be subjected to
the so-called ``connectome analysis''.

Among all the difficulties that such a complex workflow
raises \cite{jones_twenty-five_2010}, we will
address here the susceptibility-derived artifacts,
for which \Gls*{epi} schemes are specially sensitive.
As susceptibility changes at tissue interfaces,
so does the magnetic field. This inhomogeneities 
of the field translate in a highly distorted imaged 
anatomy and a significant signal  destruction of 
certain regions of the brain 
(e.g. the orbitofrontal lobe, for the proximity of the
air surrounding sinuses). This artifact has been
profoundly described, generally within the context of
functional MRI because it is usually acquired with 
\gls*{epi} as \gls*{dmri}. 

\Gls*{fmb} methods supposed the first approach to retrospective
correction \cite{jezzard_correction_1995}. They rely on one
extra acquisition (\emph{field mapping}), that probes the 
inhomogeneity of the field.
A second theory-based breed of methodologies acquire a 
map of the \acrlong*{psf} of the \gls*{epi} readouts to correct 
for the artifact \cite{robson_measurement_1997}. 
Next generation of methodologies \cite{cordes_geometric_2000,
chiou_simple_2000}, make use of the acquisition of 
extra \gls*{epi} volumes 
with specific differences that enable correction, for instance
the opposed direction of the phase encoding gradients. We will
refer to these methods as \gls*{reb}.
Finally, the last family of methodologies acquire an 
extra T2-weighted image, and use its anatomical correctness 
as reference to find the deformation map through nonlinear 
registration \cite{kybic_unwarping_2000,studholme_accurate_2000}.
\gls*{t2b} methods usually map
the T2 image to the \emph{baseline} volume or ``B0''
of \gls*{dmri}. The choice of T2 is due to the strong
similarity of intensity distribution with respect the B0.
More recent works report extensions or combined
approaches of existing techniques
\cite{andersson_how_2003,zaitsev_point_2004,%
holland_efficient_2010,andersson_comprehensive_2012}.

Even though aforementioned techniques for distortion correction
have been studied \cite{zeng_image_2002,wu_comparison_2008},
the lack of a gold-standard limits the possible benchmarking 
strategies. Recently, \cite{irfanoglu_effects_2012} raised the question
of distortion-derived impacts in tractography.
In this work, we propose an evaluation framework
using a digital phantom designed for connectivity assessment.
This framework enabled us to compare several correction 
techniques and characterize their geometrical accuracy 
and the dropout compensation. Finally, we report their 
impact on subsequent tractography and resulting connectivity
matrices.
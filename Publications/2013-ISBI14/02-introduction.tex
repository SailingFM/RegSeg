\section{INTRODUCTION}
\label{sec:intro}
In-vivo whole-brain connectivity analysis has been a
research topic of high interest for the last
5 years. 
\Gls*{dmri} can be used to probe the
orientation of fiber bundles within the brain,
and is generally acquired with an \gls*{epi} sequence.
After signal reconstruction,
tractography algorithms draw a map of the sampled 
structures.
These maps can represent the actual trajectories
of fiber bundles (deterministic tractography) or
pixel-wise probability of connection to a certain origin
(probabilistic tractography). Finally, the
information about these connections is collected
into a network matrix that can be subjected to
the so-called ``connectome analysis''.

Among all the difficulties that such a complex workflow
raises \cite{jones_twenty-five_2010}, we will
address here the susceptibility-derived artifacts,
for which \Gls*{epi} schemes are especially sensitive.
Magnetic susceptibility disturbs the magnetic field
close to tissue interfaces. This inhomogeneity
of the field translates in a highly distorted
anatomy and a significant signal destruction in
certain regions of the brain 
(e.g. the orbitofrontal lobe, for the proximity of the
air surrounding sinuses). This artifact has been
well described, generally within the context of
functional MRI which also uses \gls*{epi}.

Various approaches have been proposed to correct
for the disortions. \Gls*{fmb} methods
\cite{jezzard_correction_1995} rely on one extra
acquisition (\emph{field mapping}),
that probes the inhomogeneity of the B0 field.
A second theory-based breed of methodologies acquire a 
map of the \acrlong*{psf} of the \gls*{epi} readouts to correct 
for the artifact \cite{robson_measurement_1997}.
Another approach referred to as \gls*{reb}, acquires an
extra \gls*{epi} volume in the orthogonal or reversed
phase encoding that can be combined to remove the
geometric distortions
\cite{cordes_geometric_2000,chiou_simple_2000}.
The last family of methodologies acquires an
extra T2-weighted image, and uses it as an anatomical
reference to find the deformation map through nonlinear 
registration \cite{kybic_unwarping_2000,studholme_accurate_2000}.
These \gls*{t2b} methods usually map
the T2 image to the \emph{baseline} volume or \textit{b0}
of \gls*{dmri}. The choice of T2 is due to the strong
similarity of intensity distribution with respect the \textit{b0}.
More recent works report extensions or combined
approaches of existing techniques
\cite{andersson_how_2003,zaitsev_point_2004,%
holland_efficient_2010,andersson_comprehensive_2012}.

Even though the aforementioned techniques for distortion correction
have been studied \cite{zeng_image_2002,wu_comparison_2008},
the lack of a gold-standard limits benchmarking 
strategies. Recently, Irfanoglu et al.
\cite{irfanoglu_effects_2012} raised the question
of distortion-derived impact on tractography.
In this work, we propose an evaluation framework
using a digital phantom designed for connectivity assessment.
This framework enabled us to compare several correction 
techniques and characterize their geometrical accuracy 
and the dropout compensation. Finally, we report their 
impact on subsequent tractography and resulting connectivity
matrices.
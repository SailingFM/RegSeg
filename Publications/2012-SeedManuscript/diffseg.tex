\documentclass[a4paper,12pt]{article}
\usepackage{fullpage}

\usepackage[USenglish]{babel}
\usepackage[T1]{fontenc}
\usepackage{amsmath}
\usepackage{amssymb}
\usepackage{pslatex}
\usepackage[pdftex]{graphicx}

\usepackage{url}

%\usepackage{natbib}
%\usepackage{setspace}


\DeclareMathOperator{\tr}{tr}
\DeclareMathOperator{\argmax}{argmax}
\DeclareMathOperator{\argmin}{argmin}
\DeclareMathOperator{\diag}{diag}
\DeclareMathOperator{\dist}{dist}
\DeclareMathOperator{\const}{Const.}

\title{Subvoxel Segmentation in DW-MR Based on T1-Weighted MR Shape Priors}
\author{Oscar Esteban\thanks{Contributed equally} \and Dominique Zosso\footnotemark[1] \and Alessandro Daducci \and Meritxell Bach Cuadra \and (...) \and Jean-Philippe Thiran} 
\date{version 0.2, dz, \today}




\begin{document}
\maketitle


\section{Introduction}
Precise segmentation of diffusion-weighted MR images is a difficult task due to the very limited resolution (currently around $2.2\times 2.2\times 3 mm^3$). In brain tractography and subsequent connectivity analysis, precise segmentation of the CSF-grey-matter and white-matter-grey-matter interfaces is, however, an important task in order to achieve consistent parcellisation of the cortical surface. The diffusion-weighted acquisitions often suffer from severe distortions due to local field inhomogeneities (in particular in the anterior part of the brain and along the phase-encoded direction).

On the other hand, anatomical MR images (e.g. T1 and T2 weighted acquisitions) can be produced with substantially better resolution, relatively low distortions and well understood tissue contrast.

Currently [...]?

Here we propose to solve the segmentation problem by exploiting the anatomy easily extracted in T1-weighted images as strong shape-priors. We reformulate the segmentation problem as an inverse problem, where we look for an underlying deformation field (the distortion) mapping from T1 space into DW space, such that the structures identified in the T1 image, with high confidence, will optimally align with structures in DW space.

\section{Materials and Methods}

\subsection{T1 image and segmentations}
We have a T1-weighted (or other anatomical high-resolution and low-distortion) image of the subject, as well as reliable segmentations of the pial surface (CSF-grey-matter) and the white-matter-grey-matter interface. The segmentations are available both as surface meshs (e.g. freesurfer meshes) and as corresponding volumes (or sampling points within).

Let us denote $\{c_i\}_{i=1..N_c}$ the nodes of the white-matter-grey-matter interface, and $\{d_i\}_{i=1..N_d}$ the nodes of the pial surface. Within the respective volumes, we have sets of sample points, denoted $\{w_j\}_{j=1..N_w}$, $\{g_j\}_{j=1..N_g}$, and $\{o_j\}_{j=1..N_o}$ for the white matter, grey matter and CSF, respectively. All those are given in high-resolution T1 reference coordinates.



\subsection{DW image}
Low resolution of diffusion-weighted image. At each voxel we have a sampled ODF (or some extracted features).
Let us denote by $x$ the voxel and $f(x) = [ f_1, f_2, \ldots, f_N]^T(x)$ its associated feature vector.


\subsection{Distortion field parametrisation}
The transformation from T1 into DW reference coordinate space is achieved through a dense deformation field $u(x)$, such that for example the nodes of the w/g-interface are located in DW-space as follows:
\begin{equation}
c_i' = T\{c_i\} = c_i + u(c_i)
\end{equation} 
Since the nodes of the anatomical surfaces might lay off-grid, it is required to derive $u(x)$ from a discrete set of parameters $\{u_k\}_{k=1..K}$. Densification is achieved through a set of associated basis functions $\psi_k$ (e.g. rbf, interpolation splines):
\begin{equation}
u(x) = \sum_k \psi_k(x) u_k
\end{equation}
Consequently, the transformation writes
\begin{equation}
c_i' = T\{c_i\} = c_i + u(c_i) = c_i + \sum_k \psi_k(c_i)u_k
\end{equation} 
Note that, since $c_i$ remains constant in the DW segmentation process, the values of $\psi_k(c_i)$ can be precomputed. Also, provided compact support of the basis functions, the system remains relatively sparse.

\subsection{DW region descriptors}
Based on the region-wise samples $\{w_j\}$, $\{g_j\}$, and $\{o_j\}$, and the current estimate of the distortion $u$, we can compute ``expected samples'' of the respective regions in DW space, written $\{w_j'\}$, $\{g_j'\}$, and $\{o_j'\}$. Based on those samples, we may now estimate region descriptors of the DW features $f(x)$ of the three respective regions in DW space. In the simplest case, estimate the parameters $\mu_R$ and $\Sigma_R$, i.e. the regions' mean feature vector and covariance matrix, for each region $R\in \{w,g,o\}$.

\subsection{Active contours without edges}
Based on these Gaussian region descriptors, we propose an ACWE-like, piece-wise constant, variational image segmentation model (where the unknown is the deformation field)\cite{Chan2001}:
\begin{align}
E(u) &= \int_{w'(u)} (f-\mu_w)^T\Sigma_w^{-1}(f-\mu_w) dx\nonumber\\
&\quad +\int_{g'(u)} (f-\mu_g)^T\Sigma_g^{-1}(f-\mu_g) dx\\
&\quad +\int_{o'(u)} (f-\mu_o)^T\Sigma_o^{-1}(f-\mu_o) dx\nonumber
\end{align}
where the integral domains depend on the deformation field u. Note that minimizing this energy, $\argmin_u\{E\}$, yields the MAP estimate of a piece-wise smooth image model affected by Gaussian additive noise. This inverse problem is ill-posed \cite{Hadamard1902,Bertero1988}. In order to account for deformation field regularity and to render the problem well-posed, we include limiting and regularization terms into the energy functional \cite{Tichonov1963,Morozov1975}:
\begin{align}
E(u) &= \int_{w'(u)} (f-\mu_w)^T\Sigma_w^{-1}(f-\mu_w) dx\nonumber\\
&\quad +\int_{g'(u)} (f-\mu_g)^T\Sigma_g^{-1}(f-\mu_g) dx\\
&\quad +\int_{o'(u)} (f-\mu_o)^T\Sigma_o^{-1}(f-\mu_o) dx\nonumber\\
&\quad +\int u^T A u dx + \int \tr\{(\nabla u^T)^T B (\nabla u^T)\} dx\nonumber
\end{align}
These regularity terms ensure, that the segmenting contours in DW space are still close to their native shape in T1. In the simplest case, both $A=\alpha$ and $B=\beta$ are constant scalars, and the terms rewrite $\alpha \int  \|u\|^2 dx + \beta \int \left( \|\nabla u_x\|^2 + \|\nabla u_y\|^2 + \|\nabla u_z\|^2\right) dx$, which are familiar. Later, $A$ and $B$ might be spatially varying and/or $3\times 3$ matrices, therefore allowing to incorporate inhomogeneous and anisotropic regularization \cite{Nagel1986}.


\subsection{Operator splitting}
In order to make the Energy minimisation computationally more tractable, we propose the following operator splitting: Let us optimize the data terms and the regularity terms on separate copies of the deformation field, now called $u$ and $v$, constrained to be equal:
\begin{align}
E(u,v) &= \int_{w'(u)} (f-\mu_w)^T\Sigma_w^{-1}(f-\mu_w) dx\nonumber\\
&\quad +\int_{g'(u)} (f-\mu_g)^T\Sigma_g^{-1}(f-\mu_g) dx\\
&\quad +\int_{o'(u)} (f-\mu_o)^T\Sigma_o^{-1}(f-\mu_o) dx\nonumber\\
&\quad +\int v^T A v dx + \int \tr\{(\nabla v^T)^T B (\nabla v^T)\} dx\nonumber
\end{align}
and now
\begin{equation}
\min_{u,v} \{ E \} \quad s.t. \quad u = v
\end{equation}
In order to take the equality constraint into account, we may make use of augmented Lagrangians (a combination of Lagrangian multipliers and penalty terms on the constraint) \cite{Bertsekas1976,Glowinski1989,Nocedal2006}:
\begin{equation}
AL(u,v,\lambda,r) = E(u,v) + \langle \lambda, u-v \rangle + \frac{r}{2} \| u - v \|_2^2
\end{equation}
To solve the constraint minimization problem, we may now optimize the Augmented Lagrangian in an iterative way:
\begin{equation}
\left\lbrace \begin{array}{rcl}
u^{t+1} &=& \argmin_{u} AL(u,v^t,\lambda^t,r)\\
v^{t+1} &=& \argmin_{v} AL(u^{t+1},v,\lambda^t,r)\\
\lambda^{t+1} &=& \lambda^t + \rho(u^{t+1}-v^{t+1})
\end{array}\right.\end{equation}
where typically $0 < \rho < r$. The two subminimization problems will now be much easier to handle than the original complete problem (``divide et impera'').

\subsection{Shape gradients}
To compute the gradient-descent of the data-term domain integrals with respect to the underlying deformation field, we want to make use of shape gradients \cite{Jehan-Besson2003,Herbulot2006}. A little bit of theory is therefore in order.

Let $\Omega$ be an image domain and $\omega$ its boundary. Further, $r(x)$ is an ``arbitrary'' function over the image domain. We now derive the domain integral w.r.t. the contour evolution parameter $\tau$ (~ time):
\begin{equation}
\frac{\partial}{\partial \tau} \int_\Omega r(x) dx = \int_\Omega \frac{\partial r}{\partial \tau}(x) dx - \int_\omega r(x) \left\langle \frac{\partial\omega}{\partial\tau}, N_\omega\right\rangle dx
\end{equation}
where $\left\langle\frac{\partial\omega}{\partial\tau}, N_\omega\right\rangle$ is the projection of the boundary movement on the unit inward normal.


\subsection{Min w.r.t. $u$}
The first minimization problem optimizes the data-term. The problem is equivalent to minimizing the following energy:
\begin{align}
E(u) &= \int_{w'(u)} (f-\mu_w)^T\Sigma_w^{-1}(f-\mu_w) dx\nonumber\\
&\quad +\int_{g'(u)} (f-\mu_g)^T\Sigma_g^{-1}(f-\mu_g) dx\\
&\quad +\int_{o'(u)} (f-\mu_o)^T\Sigma_o^{-1}(f-\mu_o) dx\nonumber\\
&\quad + \langle \lambda, u-v \rangle + \frac{r}{2} \| u - v \|_2^2\nonumber
\end{align}
where we identify three instances of domain integrals of the form $\int_\Omega r(x) dx$. Optimality requires the derivative of this energy with respect to the parameters $u$ to be null. At this point, we may decide to ignore the influence of the boundary shift on the statistics of the regions (i.e. moving the boundary does not significantly impact the $\mu$ and $\Sigma$ descriptors). This means that we can drop the derivative of $r(x)$ w.r.t. contour evolution. What remains, are surface integrals at the two respective domain interfaces, $c'$ (wm/gm) and $d'$ (gm/CSF), plus the Lagrangian and penalty terms:
\begin{align}
\frac{\partial E}{\partial u_k^a} &= \int_{c'} \left[(f-\mu_g)^T\Sigma_g^{-1}(f-\mu_g) - (f-\mu_w)^T\Sigma_w^{-1}(f-\mu_w)\right]\left\langle\frac{\partial c'(s)}{\partial u_k^a}, N_{c'}(s)\right\rangle ds \nonumber\\
&\quad + \int_{d'} \left[(f-\mu_o)^T\Sigma_o^{-1}(f-\mu_o) - (f-\mu_g)^T\Sigma_g^{-1}(f-\mu_g)\right]\left\langle\frac{\partial d'(s)}{\partial u_k^a}, N_{d'}(s)\right\rangle ds \\
&\quad + \lambda_k^a + r(u_k^a - v_k^a)\nonumber\\
& = 0\nonumber
\end{align}
where $u_k^a$ is the $a$-th component of the parameter $u_k$, $a\in \{x,y,z\}$, $s$ is the surface parameter, $c'(s)$ and $d'(s)$ the corresponding points on the surfaces $c'$ and $d'$, and $N_{c'}(s)$ and $N_{d'}(s)$ the associated brainwise-inward unit normals.
Given the deformation field interpolation stated above, the boundary moves according to
\begin{equation}
\frac{\partial c'(s)}{\partial u_k^a} = \psi_k(c'(s))e_a
\end{equation}
where $e_a$ is the unit vector along direction $a$, and thus
\begin{equation}
\left\langle\frac{\partial c'(s)}{\partial u_k^a}, N_{c'}(s)\right\rangle = \psi_k(c'(s))N_{c'}^a(s)
\end{equation}
Now, we will discretize the surface integrals over $c'$ and $d'$ by simply summing over the nodes $c_i'$ and $d_i'$:
\begin{align}
\frac{\partial E}{\partial u_k^a} &= \sum_{i=1}^{N_c} \left[(f(c_i')-\mu_g)^T\Sigma_g^{-1}(f(c_i')-\mu_g) - (f(c_i')-\mu_w)^T\Sigma_w^{-1}(f(c_i')-\mu_w)\right]\psi_k(c_i')N_{c_i'}^a \nonumber\\
&\quad + \sum_{i=1}^{N_d} \left[(f(d_i')-\mu_o)^T\Sigma_o^{-1}(f(d_i')-\mu_o) - (f(d_i')-\mu_g)^T\Sigma_g^{-1}(f(d_i')-\mu_g)\right]\psi_k(d_i')N_{d_i'}^a \\
&\quad + \lambda_k^a + r(u_k^a - v_k^a)\nonumber\\
& = 0\nonumber
\end{align}
It is straightforward to solve this equation for $u_k^a$. The optimal distortion $u_k$ is found at each iteration as:
\begin{align}
u_k^{t+1} &= v_k^t - \frac{1}{r}\lambda_k^{t}\nonumber\\
&\quad - \frac{1}{r}\sum_{i=1}^{N_c} \left[(f(c_i')-\mu_g)^T\Sigma_g^{-1}(f(c_i')-\mu_g) - (f(c_i')-\mu_w)^T\Sigma_w^{-1}(f(c_i')-\mu_w)\right]\psi_k(c_i')N_{c_i'}\\
&\quad - \frac{1}{r}\sum_{i=1}^{N_d} \left[(f(d_i')-\mu_o)^T\Sigma_o^{-1}(f(d_i')-\mu_o) - (f(d_i')-\mu_g)^T\Sigma_g^{-1}(f(d_i')-\mu_g)\right]\psi_k(d_i')N_{d_i'}\nonumber
\end{align}

\subsection{Min w.r.t. $v$}

\textbf{It is important to realize that here we do not regularize the actual deformation field (i.e. after interpolation), but the underlying raw parameter field. }

For the optimization w.r.t. $v$, the relevant energy writes
\begin{align}
E(v) &= \int v^T A v dx + \int \tr\{(\nabla v^T)^T B (\nabla v^T)\} dx\\
&\quad + \langle \lambda, u-v \rangle + \frac{r}{2} \| u - v \|_2^2\nonumber
\end{align}
Let's assume the simplest, homogeneous isotropic case, $A = \alpha/2$ and $B = \beta/2$. The associated Euler-Lagrange equation is found as:
\begin{equation}
\alpha v - \beta\Delta v + rv = ru + \lambda
\end{equation}
which easily translates into Fourier domain:
\begin{equation}
v^{t+1} = \mathcal{F}^{-1}\left\{ \frac{\mathcal{F}\{ru + \lambda\}}{\mathcal{F}\{(\alpha+r)\mathcal{I}-\beta\Delta\}} \right\}
\end{equation}
where $\mathcal{I}$ denotes the identity operator.

Here, we rewrite the Laplacian as a linear combination of the identity and shift operators:
\begin{equation}
\Delta = \mathcal{S}_x^- + \mathcal{S}_x^+ + \mathcal{S}_y^- + \mathcal{S}_y^+ - 4 \mathcal{I}
\end{equation}
where $\mathcal{S}_{x,y}^{\pm}$ stands for the forward ($+$) and backward ($-$) shift operator along $x$ or $y$, respectively, of which the Fourier transform is found easily as
\begin{equation}
\mathcal{F}\{\mathcal{S}_{x,y}^{\pm}\} = e^{\pm i\omega_{x,y}},
\end{equation}
where $\omega_{x,y}$ is the normalized pulsation along $x$- and $y$-direction. Accordingly, the Fourier transform of the discrete Laplacian is found as
\begin{align}
\mathcal{F}\{\Delta\} &= e^{-i\omega_x } + e^{i\omega_x } + e^{-i\omega_y } + e^{i\omega_y } - 4\nonumber\\
&= 2\left( \cos(\omega_x) + \cos(\omega_y) - 2 \right)
\end{align}

The remaining transforms are trivial or can be computed using FFT (as in \cite{Estellers2011b}).


\subsection{Lagrangian multiplier update}
At each iteration, the Lagrangian multipliers are updated as noted before:
\begin{equation}
\lambda^{t+1} = \lambda^t + \rho(u^{t+1}-v^{t+1})
\end{equation}

\subsection{Region descriptor reestimation}
In regular intervals, i.e. after $n$ iterations, the parameters $\mu$ and $\Sigma$ of the involved regions need to be reestimated based on the shifted volumetric samples $w_j'$, $g_j'$ and $o_j'$.

\subsection{Convergence}
In order to ``fixate'' the evolution when close to convergence, it is advised to slightly increase the penalty weight $r$ at each iteration. Note that as can be seen in the above equations, $r$ governs the step-size or  leash-length at each iteration, i.e. the amount by which the new estimate $u$ may move away from the preceding $v$ and vice-versa. 


\subsection{Alternative route: Gradient descent}
If one absolutely wants to avoid the operator splitting and the augmented Lagrangians, then simple gradient descent may work as well (at least under the same simple model circumstances).

At each iteration, we update the distortion along the steepest energy descent:
\begin{equation}
\frac{\partial u_k^t}{\partial t} = -\frac{\partial E(u)}{\partial u_k^t}
\end{equation}
At this point, we interpret the parameter field $u_k$ to be a continuous function $u(x)$, sampled at the locations $x_k$, and determine the gradient-descent equation\footnote{The same assumption was being made above in the minimization of the regularity terms}:
\begin{align}
\frac{\partial u^t}{\partial t} &= - \sum_{i=1}^{N_c} \left[(f(c_i')-\mu_g)^T\Sigma_g^{-1}(f(c_i')-\mu_g) - (f(c_i')-\mu_w)^T\Sigma_w^{-1}(f(c_i')-\mu_w)\right]\psi_{c_i'}(x)N_{c_i'}\nonumber\\
&\quad -\sum_{i=1}^{N_d} \left[(f(d_i')-\mu_o)^T\Sigma_o^{-1}(f(d_i')-\mu_o) - (f(d_i')-\mu_g)^T\Sigma_g^{-1}(f(d_i')-\mu_g)\right]\psi_{d_i'}(x)N_{d_i'}\\
&\quad -\alpha u + \beta\Delta u\nonumber
\end{align}
where we have swapped $\psi_k(c_i')$ into $\psi_{c_i'}(x)$.

This gradient descent step can be efficiently tackled by discretizing the time in a forward Euler scheme, and making the right hand side semi-implicit in the regularization terms:
\begin{align}
\frac{u^{t+1}-u^t}{\tau} &= - \sum_{i=1}^{N_c} \left[(f(c_i')-\mu_g)^T\Sigma_g^{-1}(f(c_i')-\mu_g) - (f(c_i')-\mu_w)^T\Sigma_w^{-1}(f(c_i')-\mu_w)\right]\psi_{c_i'}(x)N_{c_i'}\nonumber\\
&\quad -\sum_{i=1}^{N_d} \left[(f(d_i')-\mu_o)^T\Sigma_o^{-1}(f(d_i')-\mu_o) - (f(d_i')-\mu_g)^T\Sigma_g^{-1}(f(d_i')-\mu_g)\right]\psi_{d_i'}(x)N_{d_i'}\nonumber\\
&\quad -\alpha u^{t+1} + \beta\Delta u^{t+1}
\end{align}
where the data terms remain functions of the current estimate $u^t$, i.e. all $c_i' = c_i'(u^t)$ and $d_i' = d_i'(u^t)$. Again, we propose a spectral approach to solve this implicit scheme:
\begin{equation}
u^{t+1} = \mathcal{F}^{-1}\left\{ \frac{\mathcal{F}\{u^t/\delta - \sum_{i=1}^{N_c}(\ldots) - \sum_{i=1}^{N_d}(\ldots)  \}}{\mathcal{F}\{(1/\delta+\alpha)\mathcal{I}-\beta\Delta\}} \right\}
\end{equation}
It is easily verified, that the same update can be obtained by plugging the AL-update w.r.t. $u$ into the AL-update w.r.t. $v$, and by identifying $r = 1/\delta$ (the only exception is the distortion on which the data-term is being evaluated).

\section{Experiments}
\section{Results and Discussion}
\section{Conclusions \& Outlook}
Possible complications:
\begin{itemize}
\item non-Gaussian region models (e.g. penalize entropy instead)
\item inhomogeneous, anisotropic regularization (that complicates alot!)
\item regularize dense distortion field rather than raw parameter field (in particular if number of params is limited)
\item take the region terms in the shape gradient into account
\end{itemize}

%%%% Bibliography  %%%%%%%%%%
\bibliographystyle{IEEEtran}
\bibliography{library}


\end{document}
